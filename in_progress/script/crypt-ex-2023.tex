\documentclass[11pt,a4paper]{report}

\usepackage{xcolor}
\def\farbe{blue}

\usepackage{dclecture}

\usepackage{verbatim}


\usepackage{circuitikz}

\ctikzset{
    logic ports=ieee,
    logic ports/scale=0.8,
    logic ports/fill=lightgray
}

\usetikzlibrary{arrows,shapes.gates.logic.US,shapes.gates.logic.IEC,calc}

\makeatletter

\usepackage{listings}

\definecolor{lightgray}{rgb}{0.95, 0.95, 0.95}
\definecolor{darkgray}{rgb}{0.4, 0.4, 0.4}
%\definecolor{purple}{rgb}{0.65, 0.12, 0.82}
\definecolor{editorGray}{rgb}{0.95, 0.95, 0.95}
\definecolor{editorOcher}{rgb}{1, 0.5, 0} % #FF7F00 -> rgb(239, 169, 0)
\definecolor{editorGreen}{rgb}{0, 0.5, 0} % #007C00 -> rgb(0, 124, 0)
\definecolor{orange}{rgb}{1,0.45,0.13}		
\definecolor{olive}{rgb}{0.17,0.59,0.20}
\definecolor{brown}{rgb}{0.69,0.31,0.31}
\definecolor{purple}{rgb}{0.38,0.18,0.81}
\definecolor{lightblue}{rgb}{0.1,0.57,0.7}
\definecolor{lightred}{rgb}{1,0.4,0.5}
\usepackage{upquote}
\usepackage{listings}
% CSS
\lstdefinelanguage{CSS}{
  keywords={color,background-image:,margin,padding,font,weight,display,position,top,left,right,bottom,list,style,border,size,white,space,min,width, transition:, transform:, transition-property, transition-duration, transition-timing-function},	
  sensitive=true,
  morecomment=[l]{//},
  morecomment=[s]{/*}{*/},
  morestring=[b]',
  morestring=[b]",
  alsoletter={:},
  alsodigit={-}
}

% JavaScript
\lstdefinelanguage{JavaScript}{
  morekeywords={typeof, new, true, false, catch, function, return, null, catch, switch, var, if, in, while, do, else, case, break},
  morecomment=[s]{/*}{*/},
  morecomment=[l]//,
  morestring=[b]",
  morestring=[b]'
}

\lstdefinelanguage{HTML5}{
  language=html,
  sensitive=true,	
  alsoletter={<>=-},	
  morecomment=[s]{<!-}{-->},
  tag=[s],
  otherkeywords={
  % General
  >,
  % Standard tags
	<!DOCTYPE,
  </html, <html, <head, <title, </title, <style, </style, <link, </head, <meta, />,
	% body
	</body, <body,
	% Divs
	</div, <div, </div>, 
	% Paragraphs
	</p, <p, </p>,
	% scripts
	</script, <script,
  % More tags...
  <canvas, /canvas>, <svg, <rect, <animateTransform, </rect>, </svg>, <video, <source, <iframe, </iframe>, </video>, <image, </image>, <header, </header, <article, </article
  },
  ndkeywords={
  % General
  =,
  % HTML attributes
  charset=, src=, id=, width=, height=, style=, type=, rel=, href=,
  % SVG attributes
  fill=, attributeName=, begin=, dur=, from=, to=, poster=, controls=, x=, y=, repeatCount=, xlink:href=,
  % properties
  margin:, padding:, background-image:, border:, top:, left:, position:, width:, height:, margin-top:, margin-bottom:, font-size:, line-height:,
	% CSS3 properties
  transform:, -moz-transform:, -webkit-transform:,
  animation:, -webkit-animation:,
  transition:,  transition-duration:, transition-property:, transition-timing-function:,
  }
}

\lstdefinestyle{htmlcssjs} {%
  % General design
%  backgroundcolor=\color{editorGray},
  basicstyle={\footnotesize\ttfamily},   
  frame=b,
  % line-numbers
  xleftmargin={0.75cm},
  numbers=left,
  stepnumber=1,
  firstnumber=1,
  numberfirstline=true,	
  % Code design
  identifierstyle=\color{black},
  keywordstyle=\color{blue}\bfseries,
  ndkeywordstyle=\color{editorGreen}\bfseries,
  stringstyle=\color{editorOcher}\ttfamily,
  commentstyle=\color{brown}\ttfamily,
  % Code
  language=HTML5,
  alsolanguage=JavaScript,
  alsodigit={.:;},	
  tabsize=2,
  showtabs=false,
  showspaces=false,
  showstringspaces=false,
  extendedchars=true,
  breaklines=true,
  % German umlauts
  literate=%
  {Ö}{{\"O}}1
  {Ä}{{\"A}}1
  {Ü}{{\"U}}1
  {ß}{{\ss}}1
  {ü}{{\"u}}1
  {ä}{{\"a}}1
  {ö}{{\"o}}1
}
%
\lstdefinestyle{py} {%
language=python,
literate=%
*{0}{{{\color{lightred}0}}}1
{1}{{{\color{lightred}1}}}1
{2}{{{\color{lightred}2}}}1
{3}{{{\color{lightred}3}}}1
{4}{{{\color{lightred}4}}}1
{5}{{{\color{lightred}5}}}1
{6}{{{\color{lightred}6}}}1
{7}{{{\color{lightred}7}}}1
{8}{{{\color{lightred}8}}}1
{9}{{{\color{lightred}9}}}1,
basicstyle=\footnotesize\ttfamily, % Standardschrift
numbers=left,               % Ort der Zeilennummern
%numberstyle=\tiny,          % Stil der Zeilennummern
%stepnumber=2,               % Abstand zwischen den Zeilennummern
numbersep=5pt,              % Abstand der Nummern zum Text
tabsize=4,                  % Groesse von Tabs
extendedchars=true,         %
breaklines=true,            % Zeilen werden Umgebrochen
keywordstyle=\color{blue}\bfseries,
frame=b,
commentstyle=\color{brown}\itshape,
stringstyle=\color{editorOcher}\ttfamily, % Farbe der String
showspaces=false,           % Leerzeichen anzeigen ?
showtabs=false,             % Tabs anzeigen ?
xleftmargin=17pt,
framexleftmargin=17pt,
framexrightmargin=5pt,
framexbottommargin=4pt,
%backgroundcolor=\color{lightgray},
showstringspaces=false,      % Leerzeichen in Strings anzeigen ?
}%
%
\makeatother



%%% Fancy Header and Footer
\renewcommand{\headrule}{\vbox to 0pt{\hbox to\headwidth{\color{\farbe}\rule{\headwidth}{1pt}}\vss}}
\pagestyle{fancy} %eigener Seitenstil
\fancyhf{} %alle Kopf- und Fusszeilenfelder bereinigen
\fancyhead[C]{Computer Science} %Kopfzeile mitte
%\fancyhead[R]{\includegraphics[width=0.2cm]{x.png}}
\fancyfoot[C]{\thepage}


\newcommand{\bfb}[1]{{\bf \color{blue} #1}}




\begin{document}


\begin{ex}
Encrypt \verb|THE QUICK BROWN FOX| using a Caesar shift of 17.
\end{ex}
\sol{
\texttt{KYV HLZTB SIFNE WFO}
}


\begin{ex}
Eve intercepted this message: \verb|N QNPJ HNUMJWX.| Figure out how to break it to get Alice’s message.
\end{ex}
\sol{
Shift: 5 \\
Plaintext: \texttt{I LIKE CIPHERS.}
}


\begin{ex}
Use the website \url{https://cryptii.com/} to decode the following message:
\begin{verbatim}
Pyebc mybok xncof oxiok bckqy yebpk drobc lbyeq rdpyb drezy xdrsc
myxds xoxd, kxogx kdsyx, myxm osfon sxVsl obdi, kxnno nsmkd ondyd
rozby zycsd syxdr kdkvv woxkb ombok donoa ekv.
\end{verbatim}
\end{ex}
\sol{
Shift: 10 \\
\texttt{Fours corea ndsev enyea rsago ourfa thers broug htfor thupo nthis
conti nent, anewn ation, conc eived inLib erty, andde dicat edtot
hepro posit ionth atall menar ecrea tedeq ual.}  \\
or with the proper spacing:  \\
\texttt{Four score and seven years ago our fathers brought forth, upon this continent, a new nation, conceived in liberty, and dedicated to the proposition that all men are created equal.}
}

\begin{ex}
Alice tries to make the Caesar code better by using multiple Caesar shifts. Her argument is: "If I use 50 Caesar shifts,  then the keyspace will be $26^{50}$ -- which is much larger than the standard keyspace of 26 for the Caesar code.  This means the code will be harder to crack." 

Is her argument sound?
\end{ex}
\sol{
No, two Caesar shifts can always be combined to a single shift. e.g. a shift of $7$ followed by a shift of $17$ is identical to a shift of $24$. Also a shift greater than $26$ is identical to the same shift modulo $26$. e.g. a shift of $37$ is identical to a shift of $11$.
}




\begin{ex}
Given the following One Time Pad: 
\begin{verbatim}
74288, 12127, 11380, 68612, 71750, 80865, 45593, 84621, 37271,
22781, 46647, 18730, 62697, 23437, 36413, 51216, 65713, 74749,
44834, 89655
\end{verbatim}
encrypt the message \verb|TOP SECRET| using a standard alphabet replacement (a = 1, b = 2, etc.) e.g. the T corresponds to the value $20$. So we take the first entry in the OTP \verb|74288| and add 20: \verb|74308|. This is now the encryption for our first letter etc.
\end{ex}
\sol{
\texttt{74308, 12142, 11396, 68631, 71755, 80868, 45611, 84626, 37291}
}


\begin{ex}
Given the following One Time Pad: 
\begin{verbatim}
74288, 12127, 11380, 68612, 71750, 80865, 45593, 84621, 37271,
22781, 46647, 18730, 62697, 23437, 36413, 51216, 65713, 74749,
44834, 89655
\end{verbatim}
encrypt the message \verb|TOP SECRET| using the ASCII table. e.g. the T corresponds to the value $84$. So we take the first entry in the OTP \verb|74288| and add 84: \verb|74372|. This is now the encryption for our first letter etc.
\end{ex}
\sol{
\texttt{74372, 12206, 11460, 68695, 71819, 80932, 45675, 84653, 37355}
}

\begin{ex}
blank
\end{ex}


\begin{ex}
Encrypt the following text using a Vigenère Cipher with the keyword \verb|SUPERMAN|:
\begin{verbatim}
I LOVE MATH
\end{verbatim}

\end{ex}
\sol{
\texttt{Q RZRN AAGP}
}

\begin{ex}
Encrypt the following text using a Vigenère Cipher with the keyword \verb|SUPERMAN| using the online tool \url{https://cryptii.com/}:
\begin{verbatim}
To be, or not to be, that is the question:
Whether 'tis nobler in the mind to suffer
The slings and arrows of outrageous fortune,
Or to take arms against a sea of troubles
And by opposing end them. To die - to sleep,
No more; and by a sleep to say we end
The heart-ache and the thousand natural shocks
That flesh is heir to: 'tis a consummation
Devoutly to be wish'd. To die, to sleep;
\end{verbatim}
\end{ex}


\begin{ex}
Use the online tool \url{https://www.101computing.net/frequency-analysis/} to run a frequency analysis on the ciphertext. Then decrypt the hidden message.
\end{ex}



\begin{ex}
Write out the message
\verb|BEWARE THE IDES OF MARCH.|
in rows of length 5 and read out the columns. Give the ciphertext.
\end{ex}
\sol{
\texttt{BEDM ETEA WHSR AEOC RIFH}
}

\begin{ex}
Decrypt the following message using length 4: \verb|INLRREINVCFIIAOA|.
\end{ex}
\sol{
\texttt{Irvine, California}
}

\begin{ex}
Eve intercepted the following message. Help her break it: \verb|HYDAMAPOYPNZ|.
\end{ex}
\sol{
Rows of three: \texttt{HAPPY MONDAY} with a filler character, that can be ignored.
}







\begin{ex}
Clone the code from the repository \url{https://github.com/olidec/caesar}. Read and try to understand the code that encrypts a text with the Caesar cipher. We will talk through the code together.
\end{ex}

\begin{ex}
Write a new function \verb|decryptCaesar()| that takes a ciphertext as input and returns the plaintext.
\end{ex}
\sol{
We can replace the \texttt{+s} with \texttt{-s}, however we must ensure that the character lies within the range of the standard alphabet. We do this by adding 26 to our result before calculating the modulo -- this will ensure that we always decrypt a letter with a letter.
}

\begin{ex}
Write code that encrypts plaintext with the atbash cipher. 
\end{ex}
\sol{
Here we replace the caesar-shift with \texttt{j = 25-j}. Since the original character codes are between 0 and 25 after the 'shift' back, we do not have to worry about exiting the letters.
}

\begin{ex}
(For Specialists) \\
Use the code from the \verb|encryptCaesar| function to encode a text using a Vigenère cipher. 

Option 1: use a fixed key \\
Option 2: add an input field for the key
\end{ex}

\newpage

\subsection*{Solutions}
\printcursols


\end{document}
