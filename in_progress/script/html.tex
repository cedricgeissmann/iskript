\documentclass[english,11pt,a4paper]{report}

\usepackage{xcolor}
\def\farbe{blue}

\usepackage{dclecture}

\usepackage{color}
\definecolor{editorGray}{rgb}{0.95, 0.95, 0.95}
\definecolor{editorOcher}{rgb}{1, 0.5, 0} % #FF7F00 -> rgb(239, 169, 0)
\definecolor{editorGreen}{rgb}{0, 0.5, 0} % #007C00 -> rgb(0, 124, 0)
\usepackage{upquote}
\usepackage{listings}
\lstdefinelanguage{JavaScript}{
  morekeywords={typeof, new, true, false, catch, function, return, null, catch, switch, var, if, in, while, do, else, case, break},
  morecomment=[s]{/*}{*/},
  morecomment=[l]//,
  morestring=[b]",
  morestring=[b]'
}

\lstdefinelanguage{HTML5}{
        language=html,
        sensitive=true, 
        alsoletter={<>=-},
        otherkeywords={
        % HTML tags
        <html>, <head>, <title>, </title>, <meta, />, </head>, <body>,
        <canvas, \/canvas>, <script>, </script>, </body>, </html>, <!, html>, <style>, </style>, ><
        },  
        ndkeywords={
        % General
        =,
        % HTML attributes
        charset=, id=, width=, height=,
        % CSS properties
        border:, transform:, -moz-transform:, transition-duration:, transition-property:, transition-timing-function:
        },  
        morecomment=[s]{<!--}{-->},
        tag=[s]
}

\lstset{%
    % Basic design
    backgroundcolor=\color{editorGray},
    basicstyle={\small\ttfamily},   
    frame=l,
    % Line numbers
    xleftmargin={0.75cm},
    numbers=left,
    stepnumber=1,
    firstnumber=1,
    numberfirstline=true,
    % Code design   
    keywordstyle=\color{blue}\bfseries,
    commentstyle=\color{darkgray}\ttfamily,
    ndkeywordstyle=\color{editorGreen}\bfseries,
    stringstyle=\color{editorOcher},
    % Code
    language=HTML5,
    alsolanguage=JavaScript,
    alsodigit={.:;},
    tabsize=2,
    showtabs=false,
    showspaces=false,
    showstringspaces=false,
    extendedchars=true,
    breaklines=true,        
    % Support for German umlauts
    literate=%
    {Ö}{{\"O}}1
    {Ä}{{\"A}}1
    {Ü}{{\"U}}1
    {ß}{{\ss}}1
    {ü}{{\"u}}1
    {ä}{{\"a}}1
    {ö}{{\"o}}1
}

\begin{document}
\section{Learning HTML}

In this chapter we will go through the basics of HTML and take our first steps in the making of a website.

\subsection{What is HTML?}

HTML is an abbreviation for \emph{Hypertext Markup Language}. Hypertext refers to the fact that we can link multiple such files together (so-called hyperlinks). A markup language is a way of writing text where we give the output application instructions on how it should be formatted. 

The only text editor you probably know is 'Word' -- a WYSIWYG (what you see i what you get) editor. This automatically shows you what the final version looks like. However, it may be difficult to coordinate styles and layouts over larger documents or even across multiple files. For this reason webpages are written using the markup language HTML. Other markup languages exist and are used in various places (more on that later).

\subsection{The Structure of a Website}
A modern website usually consists of three main parts: 
\begin{itemize}
    \item An HTML-file containing the content and layout of the webpage.
    \item A CSS-file containing information on how to \emph{style} the webpage.
    \item A JavaScript-file to add interactivity.
\end{itemize}

We will start by looking at HTML, then we will add CSS and finally we will look at how to make pages interactive using JavaScript. 

\subsection{The Structure of a HTML file}
An HTML file consists of two main parts: a head and a body. The head contains metainformation about the webpage, whereas the body contains all of the actual content.

Within the head and the body are HHTML \emph{elements}. These are the basic building blocks of HTML code and usually have the following structure: 
\begin{itemize}
    \item An \emph{opening tag} (e.g. \lstinline|<p>|)
    \item \emph{Content} (e.g. Hello)
    \item A \emph{closing tag} (e.g. \lstinline|</p>|)
\end{itemize}

So the HTML element for a paragraph \lstinline|p| would look like this:
\begin{lstlisting}
    <p> Hello </p>
\end{lstlisting}
Note that it does not matter whether you write this on one or more lines. Everything between the opening and closing tags will be rendered accordingly:
\begin{lstlisting}
    <p> 
        Hello 
    </p>
\end{lstlisting}

\subsection{A Tutorial}
The following link leads to a tutorial on HTML. \\
\url{https://www.w3schools.com/html/html_intro.asp} 

You do not have to look at all sections, but the following are roughly the ones that will be tested in the short quiz:
\begin{itemize}
    \item HTML Introduction
    \item HTML Basic
    \item HTML Elements
    \item HTML Attributes
    \item HTML Headings
    \item HTML Paragraphs
    \item HTML Styles
    \item HTML Formatting
    \item HTML Colors
    % \item HTML CSS
    \item HTML Links
    \item HTML Images
    \item HTML Page Title
    \item HTML Lists
    \item HTML Block \& Inline
    \item HTML Classes
    \item HTML Id
    \item HTML Head
    \item HTML Layout
    \item HTML Style Guide
\end{itemize}

\newpage
\subsection{A Basic HTML File}
In order for your web browser to properly render your webpage, you have to follow certain standards and norms. This ensures that the webpage looks the same for all users. A very basic HTML file looks like this:

\begin{lstlisting}
<!DOCTYPE html>
<html>
<head>
    <meta charset="UTF-8" />
    <title>title</title>
</head>
<body>
    
</body>
</html>
\end{lstlisting}

Let's break this down:
\begin{itemize}
    \item The \lstinline|!DOCTYPE| tag tells the browser that what follows is written in HTML.
    \item The opening and closing \lstinline|html| tags include all of the HTML code.
    \item The \lstinline|head| tags enclose metadata about the webpage -- in this case the character set (\lstinline|charset|) used (worry about that later) and the title of the webpage (between the \lstinline|title| tags).
    \item Everything between the \lstinline|body| tags will be the content of the webpage.
\end{itemize}

\subsection{Workflow for Getting Started}
In general, start each new project in a new folder. This will help you keep an overview over all the different projects you may have. Our standard code editor will be Visual Studio Code and I will give you the basics for writing code with this editor. If you use a different editor, you will have to work out the details on your own.

\renewcommand{\labelenumi}{\arabic{enumi}.}
\begin{enumerate}
    \item When you open up VSCode for the first time you will probably see something like this:
    \centfig{0.5}{welcome}
    \item Under the file menu select 'Open Folder'.
    \centfig{0.5}{open-folder2}
    \item In the dialog box select 'New Folder'.
    \centfig{0.5}{new-folder}
    \item Write the name of your new folder, click 'Create'  and then click 'Open'.
    \centfig{0.5}{newproject}
    \item You should now see an empty workspace. Show the file browser by clicking on the 'pages' in the top left corner. \centfig{0.5}{file-explorer}
    \item Add a new file by clicking on the 'New File' button at the top of the file explorer.
    \centfig{0.5}{new-file}
    \newpage
    \item Name your new file \verb|index.html|.
    \centfig{0.5}{index.png}
    \item Within the text area of your empty file type \verb|html:5| and push 'return'. This should give you a basic HTML template to work with. Hit \verb|control-s| (\verb|command-s| on Mac) to save your work.
    \centfig{0.5}{html.png}
    \centfig{0.5}{html-template}
    \newpage
    \item Click on the 'Source Control' Panel. Click 'Publish to Github'. The first time you do this, you will be asked to verify your identity.
    \centfig{0.5}{source-control}
    \item If you have done everything correctly, your 'Source Control' panel should look like this.
    \centfig{0.5}{source-control2}
\end{enumerate}

\subsection{Basic Workflow for using Github and VSCode}
\begin{enumerate}
    \item While editing a file and before you save the changes, VSCode will tell you with a small dot in the file tab.
    \centfig{0.5}{edit}
    \item After saving (\verb|control-s| or \verb|command-s|), the 'Source Control' panel will indicate that there are changes.
    \centfig{0.5}{save}
    \item Click on the source control panel and you will see all changes that have not yet been passed to the version control.
    \centfig{0.35}{switch-to-source}
    \item Click on the small '+' at the top next to the word 'Changes' to \emph{stage} all your changes. (Note: \emph{staging} is a part of the git workflow that is not really important unless you are working on huge projects.)
    \centfig{0.5}{stage2}
    \item Next write a text describing the changes that you made. If you ever have to undo something later on this text will help you to find the correct version.
    \centfig{0.5}{commit-message}
    \newpage
    \item After clicking 'Commit' you should now see something like this.
    \centfig{0.5}{commit2}
    \item Click on 'Sync Changes' to save your files to Github. You now have an automatic online backup of your code and your source control panel should look like this.
    \centfig{0.5}{push2}
\end{enumerate}

\newpage
\subsection{Basic HTML Commands}
You should know how and when to use the following HTML tags.
\begin{itemize}
    \item \verb|<html>...</html>|
    \item \verb|<head>...</head>|
    \item \verb|<body>...</body>|
    \item \lstinline|<p>...</p>|
    \item \lstinline|<br>|
    \item \lstinline|<h1>...</h1>| through \lstinline|<h6>...</h6>|
    \item \lstinline|<div>...</div>|
    \item \lstinline|<span>...</span>|
    \item \lstinline|<ul>...</ul>|
    \item \lstinline|<ol>...</ol>|
    \item \lstinline|<li>...</li>|
    \item \verb|<a href="...">...</a>|
    \item \verb|<img src="..."/>|
    \item \verb|<script>...</script>|
\end{itemize}

\subsection{HTML Attributes}
HTML attributes provide additional information about HTML elements.

\begin{itemize}
	\item All HTML elements can have attributes
    \item Attributes provide additional information about elements
    \item Attributes are always specified in the start tag
    \item Attributes usually come in name/value pairs like: name="value"
\end{itemize}

You should know how and when to use the following attributes:
\begin{itemize}
    \item \verb|src|
    \item \verb|alt|
    \item \verb|width|
    \item \verb|height|
    \item \verb|href|
    \item \verb|target|
    \item \verb|id|
    \item \verb|class|
    \item \verb|style|
\end{itemize}

\subsection{Short Quiz}
When you feel that you understand the basics, you can solve a short quiz on moodle that will contribute points to your final grade. The test is multiple choice and you can only take it once. 



\end{document}