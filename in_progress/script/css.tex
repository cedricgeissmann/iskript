\documentclass[english,11pt,a4paper]{report}

\usepackage{xcolor}
\def\farbe{blue}

\usepackage{dclecture}

\usepackage{color}
\definecolor{editorGray}{rgb}{0.95, 0.95, 0.95}
\definecolor{editorOcher}{rgb}{1, 0.5, 0} % #FF7F00 -> rgb(239, 169, 0)
\definecolor{editorGreen}{rgb}{0, 0.5, 0} % #007C00 -> rgb(0, 124, 0)
\usepackage{upquote}
\usepackage{listings}
\lstdefinelanguage{JavaScript}{
  morekeywords={typeof, new, true, false, catch, function, return, null, catch, switch, var, if, in, while, do, else, case, break},
  morecomment=[s]{/*}{*/},
  morecomment=[l]//,
  morestring=[b]",
  morestring=[b]'
}

\lstdefinelanguage{HTML5}{
        language=html,
        sensitive=true, 
        alsoletter={<>=-},
        otherkeywords={
        % HTML tags
        <html>, <head>, <title>, </title>, <meta, />, </head>, <body>,
        <canvas, \/canvas>, <script>, </script>, </body>, </html>, <!, html>, <style>, </style>, ><
        },  
        ndkeywords={
        % General
        =,
        % HTML attributes
        charset=, id=, width=, height=,
        % CSS properties
        border:, transform:, -moz-transform:, transition-duration:, transition-property:, transition-timing-function:
        },  
        morecomment=[s]{<!--}{-->},
        tag=[s]
}

\lstset{%
    % Basic design
    backgroundcolor=\color{editorGray},
    basicstyle={\small\ttfamily},   
    frame=l,
    % Line numbers
    xleftmargin={0.75cm},
    numbers=left,
    stepnumber=1,
    firstnumber=1,
    numberfirstline=true,
    % Code design   
    keywordstyle=\color{blue}\bfseries,
    commentstyle=\color{darkgray}\ttfamily,
    ndkeywordstyle=\color{editorGreen}\bfseries,
    stringstyle=\color{editorOcher},
    % Code
    language=HTML5,
    alsolanguage=JavaScript,
    alsodigit={.:;},
    tabsize=2,
    showtabs=false,
    showspaces=false,
    showstringspaces=false,
    extendedchars=true,
    breaklines=true,        
    % Support for German umlauts
    literate=%
    {Ö}{{\"O}}1
    {Ä}{{\"A}}1
    {Ü}{{\"U}}1
    {ß}{{\ss}}1
    {ü}{{\"u}}1
    {ä}{{\"a}}1
    {ö}{{\"o}}1
}

\begin{document}
\section{Learning CSS}

\subsection{The Box Model}

\begin{flushleft}
The Box Model in CSS describes how elements occupy space on a webpage. All elements are divided into 4 areas.
\end{flushleft}
\begin{itemize}
    \item \texttt{margin}: The outer margin, i.e., the distance from other elements.
    \item \texttt{border}: The border of an element. It is usually not visible but can be used to outline elements.
    \item \texttt{padding}: The inner margin, i.e., the distance from the content to the border.
    \item \texttt{content}: The content of the element. This is usually specified in width and height.
\end{itemize}
\begin{flushleft}
To apply the Box Model most simply, you should always include the following CSS rule:
\end{flushleft}
\begin{verbatim}
* {
  box-sizing: border-box;
}
\end{verbatim}
\begin{flushleft}
\textbf{Example: Task - Playing with the Box Model}
\end{flushleft}
Create an element with the ID \textbf{box}. Experiment with the following CSS code until you understand what each part does.
\begin{verbatim}
#box {
  background-color: cyan;
  border: 2px solid black;
  margin: 50px;
  padding: 10px;
  width: 300px;
  height: 70px;
}
\end{verbatim}

\subsubsection{Specific Margins}
\begin{flushleft}
You can also address individual parts of the margins and do not have to use the same distances for all.
\end{flushleft}
\textbf{Example: Task - Colored Border on the Left Side}
With \textbf{border-left}, you can target only the left border. Create a class \textbf{danger} that has a red border of 5 pixels on the left side.
\begin{flushleft}
You can address the individual parts of the margin as follows, and the others are analogous:
\end{flushleft}
\begin{itemize}
    \item \textbf{margin-left}
    \item \textbf{margin-right}
    \item \textbf{margin-top}
    \item \textbf{margin-bottom}
\end{itemize}

\newpage

\subsection{Summary}

You should know how to use basic CSS selectors and some basic modifiers:
\begin{itemize}
  \item \verb|*|
  \item \verb|tagname|
  \item \verb|.classname|
  \item \verb|#idname|
  \item \verb|selector:hover|
\end{itemize}

You should know the following CSS properties and how to apply them:
\begin{itemize}
  \item \verb|float|
  \item \verb|margin|
  \item \verb|border|
  \item \verb|padding|
  \item \verb|width|
  \item \verb|height|
  \item \verb|color|
  \item \verb|background-color|
  \item \verb|font-size|
  \item \verb|font-weight|
  \item \verb|font-style|
  \item \verb|text-align|
\end{itemize}

In particular, you should be able to:
\begin{itemize}
  \item Set colors for text and backgrounds (using color names and rgb codes).
  \item Layout divisions and images on the page e.g. centering on a page.
  \item Use spacing to separate elements.
  \item Use your head.
\end{itemize}




\end{document}