\documentclass[11pt,a4paper]{report}

\usepackage{xcolor}
\usepackage[ngerman]{babel}
\def\farbe{blue}

\usepackage{dclecture}

\usepackage{listings}
\lstset{language=Python}


\definecolor{codegreen}{rgb}{0,0.6,0}
\definecolor{codegray}{rgb}{0.9,0.9,0.9}
\definecolor{codepurple}{rgb}{0.58,0,0.82}
\definecolor{backcolour}{rgb}{0.95,0.95,0.92}

\lstdefinestyle{mystyle}{
%	morekeywords={forward,turn},
    backgroundcolor=\color{codegray},
    commentstyle=\color{codegreen},
%    keywordstyle=\color{codegreen},
    numberstyle=\tiny\color{gray},
    stringstyle=\color{codepurple},
    basicstyle=\footnotesize,
    identifierstyle=\color{blue},
    stringstyle=\color{orange},
    breakatwhitespace=false,
    breaklines=true,
    captionpos=b,
    keepspaces=true,
    numbers=left,
    numbersep=5pt,
    showspaces=false,
    showstringspaces=false,
    showtabs=false,
    tabsize=2
}

\lstset{style=mystyle}




%%% Fancy Header and Footer
\renewcommand{\headrule}{\vbox to 0pt{\hbox to\headwidth{\color{\farbe}\rule{\headwidth}{1pt}}\vss}}
\pagestyle{fancy} %eigener Seitenstil
\fancyhf{} %alle Kopf- und Fusszeilenfelder bereinigen
\fancyhead[C]{Computer Science} %Kopfzeile mitte
%\fancyhead[R]{\includegraphics[width=0.2cm]{x.png}}
\fancyfoot[C]{\thepage}


\title{Informatik Gruppenarbeit 1MZ}
\date{Herbst 2022}



\begin{document}
\maketitle

\newpage
\section{Erste Schritte}

In den nächsten Wochen werden Sie in Gruppen an einem Programmierprojekt
arbeiten. Sie werden dabei ein interaktives Spiel entwickeln. Dabei werden Sie
eine neue Umgebung (\verb|pygame|) kennenlernen, und Ihr bereits vorhandenes
Wissen sowie neues anwenden, um ein tolles kreatives Spiel zu erstellen.

\subsection{Gruppen}

Im ersten Schritt werden Sie Gruppen bilden. Eine Gruppe besteht aus 3 bis 4
Personen. Es kann hilfreich sein die Gruppen so einzuteilen das M und Z
gemischt sind, damit arbeiten Sie im Praktikum in kleineren Gruppen, jedoch
haben Sie öfters die Gelegenheit direkt Fragen zu stellen.

\subsection{Installieren der Werkzeuge}

Um an dem Projekt zu arbeiten müssen Sie einige Werkzeuge bei Ihnen lokal auf
dem Computer installieren. Einige Werkzeuge haben Sie vermutlich bereits
installiert. Ein Teil von Ihnen hat bereits alle Tools installiert und kann dem
Rest der Gruppe bei der Installation helfen.

\begin{itemize}
\item Damit Sie Versionskontrolle für die Zusammenarbeit nutzen können, und
    damit wir überprüfen können wer wieviel beigetragen hat, müssen Sie
    \verb|Git| lokal auf Ihrem Computer installieren.

    \begin{center}
        \url{https://git-scm.com/downloads}
    \end{center}

\item Damit Sie auf Ihrem Rechner mit \verb|Python| und \verb|Pygame| arbeiten können, müssen
    Sie \verb|Python 3.6| oder neuer installieren.

        \begin{center}
            \url{https://www.python.org/}
        \end{center}

\item Als Programmierumgebung verwenden wir \verb|Visual Studio Code|. Das
    haben Sie bereits installiert.

        \begin{center}
            \url{https://code.visualstudio.com/}
        \end{center}

\item Damit Sie in \verb|Visual Studio Code| so arbeiten können wie Sie sich
    das gewohnt sind, müssen Sie noch 2 Extensions installieren:
\begin{itemize}
\item Git Graph (das von \emph{mhutchie})
\item Python
\end{itemize}
\item Sie werden auch noch Python-Module installieren müssen.

\end{itemize}

\subsection{Installieren von Modulen in Python}

Um ein Modul in Python zu installieren, müssen Sie in VSCode eine Kommandozeile
(Terminal) öffnen. Das können Sie ganz einfach machen indem Sie das \verb|Terminal|
Menu anwählen, und dann den Punkt \verb|New Terminal| auswählen.
Im Terminal müssen Sie dann einen der folgenden Befehle eingeben. Der Befehl
den Sie brauchen hängt von Ihrem Betriebssystem ab und welche Version von
Python Sie installiert haben. Für Windows funktioniert meist der erste Befehl,
und bei Mac ist es der zweite. Sollten diese Befehle nicht gehen, verwenden Sie
die anderen und Fragen Sie nach bei Problemen.

\begin{itemize}
    \item \verb|py -m pip install pygame|
    \item \verb|python3 -m pip install pygame|
    \item \verb|pip install pygame|
    \item \verb|python -m pip install pygame|
\end{itemize}

\newpage

\section{Projektbeschreibung}

\subsection{Ziele} 

Ziel des Projektes ist es ein interaktives Spiel mit der Spieleengine
\verb|pygame| zu entwickeln. Der Fokus dabei liegt auf der Entwicklung eines
Einzelspielerspiels. Mehrspielerspiele können auch entwickelt werden, diese
sollten aber am gleichen Computer stattfinden.

Einige Beispiele zur Inspiration:
\begin{itemize}
\item Jump \& Run
\item Top-Down RPG
\item Rundenbasiertes Karten oder Brettspiel
\item Capture the Flag
\item Worms
\item Lemmings
\item etc.
\end{itemize}

\subsection{Ablauf}
Das Projekt dauert von der Woche 11 bis zur Woche 14. Die Abschlusspräsentation
mit Demo des Spiels wird am 26.04. stattfinden. Die erste Woche sollte dafür
verwendet werden einen Plan für das Spiel zu erstellen und sich mit
\verb|Pygame| vertraut zu machen. Ungefähr in der Hälfte des Projektes sollten
Sie eine minimale Version von Ihrem Spiel haben, die bereits einige
Basisfeatures implementiert. Die finale Version des Spiel muss spielbar sein.

Für die Dokumentation des Spiels (worum geht es, wie wird es gespielt, wer hat
es entwickelt, wo ist es verfügbar) schreiben Sie eine eigene Webseite (dies
muss kein Webserver sein HTML und CSS reichen aus).

\subsubsection{Milensteine}
\begin{description}
    \item[am 15.03.2022] Eine kurze Präsentation (5 Minuten) über die Idee
        Ihres Spiels. Die Klasse wird da die Gelegenheit haben Fragen zu
        stellen und Anregungen zu geben.
\item[um den 31.03.2022] In einem persönlichen Treffen stellen Sie mir Ihr
    Projekt vor, was Sie bis jetzt gemacht haben und woran Sie noch arbeiten
        werden.
    \item[am 26.04.2022] Finale Version (inklusive Webseite) muss via Github
        abgegeben werden. Es findet eine Präsentation mit Demo vor der Klasse
        statt. Auftretende Fragen zum Projekt sollen beantwortet werden.
\end{description}

\subsection{Benotung}

Sie werden eine Note als Gruppe für das Projekt erhalten. Faktoren die bei der
Benotung berücksichtigt werden sind: Haben Sie die Zeit sinnvoll eingeteilt?
Hat jeder in der Gruppe zum Projekt beigetragen? Machen Sie Fortschritt? Wie
wurde das Feedback von Meilenstein 2 umgesetzt? Funktioniert Ihr Spiel? Gibt es
Bugs oder nicht fertige Features? Ist die Dokumentation vollständig und
verständlich? \dots

Die Note ist für alle Gruppenmitglieder die gleiche. Treten jedoch starke
Abweichungen in den Beiträgen auf (git commits) behalten wir uns das Recht vor
für einzelne Gruppenmitglieder eine separate Note zu setzen.

\newpage

\subsection{Ressourcen}
Sie Hauptressource die Sie verwenden werden ist die Webseite von \verb|Pygame|:
\begin{center}
    \url{https://www.pygame.org/news}
\end{center}

Hier finden Sie Beispiele und Dokumentation zu \verb|Pygame|.

Zusätzlich können Sie im Internet viele Ressourcen finden wie man mit
\verb|Pygame| arbeitet. Zum Beispiel diese Playliste:

\begin{center}
\url{https://www.youtube.com/playlist?list=PLkkm3wcQHjT7gn81Wn-e78cAyhwBW3FIc}
\end{center}

Die Playliste ist eine komplette Anleitung wie Sie ein Top-Down RPG entwickeln
können. Es werden jedoch Konzepte verwendet die Sie noch nicht kennen. Dennoch
sollte es möglich sein der Anleitung zu folgen.

Bei Fragen und Problemen können Sie mich natürlich jederzeit kontaktieren.

\end{document}
