\documentclass[12pt,a4paper]{report}

\usepackage{xcolor}
\def\farbe{blue}

\usepackage{dcsls}

\usepackage{circuitikz}

\ctikzset{
    logic ports=ieee,
    logic ports/scale=0.8,
    logic ports/fill=lightgray
}

\usetikzlibrary{arrows,shapes.gates.logic.US,shapes.gates.logic.IEC,calc}


\usepackage{listings}
\lstset{language=Python}


\definecolor{codegreen}{rgb}{0,0.6,0}
\definecolor{codegray}{rgb}{0.9,0.9,0.9}
\definecolor{codepurple}{rgb}{0.58,0,0.82}
\definecolor{backcolour}{rgb}{0.95,0.95,0.92}

\lstdefinestyle{mystyle}{
%	morekeywords={forward,turn},
    backgroundcolor=\color{codegray},
    commentstyle=\color{codegreen},
%    keywordstyle=\color{codegreen},
    numberstyle=\tiny\color{gray},
    stringstyle=\color{codepurple},
    basicstyle=\footnotesize,
    identifierstyle=\color{blue},
    stringstyle=\color{orange},
    breakatwhitespace=false,
    breaklines=true,
    captionpos=b,
    keepspaces=true,
    numbers=left,
    numbersep=5pt,
    showspaces=false,
    showstringspaces=false,
    showtabs=false,
    tabsize=2
}

\lstset{style=mystyle}

\setlength{\headheight}{25pt}


%%% Fancy Header and Footer
\renewcommand{\headrule}{\vbox to 0pt{\hbox to\headwidth{\color{\farbe}\rule{\headwidth}{1pt}}\vss}}
\pagestyle{fancy} %eigener Seitenstil
\fancyhf{} %alle Kopf- und Fusszeilenfelder bereinigen
\fancyhead[C]{Artificial Intelligence and Ethics} %Kopfzeile mitte
%\fancyhead[R]{\includegraphics[width=0.2cm]{x.png}}
\fancyfoot[C]{\thepage}


\newcommand{\bfb}[1]{{\bf \color{blue} #1}}




\begin{document}

\section*{Assignment 1: How does a neural network work?}

\subsection*{The 3-2-1 Method}
For the following assignment you will be asked to use the 3-2-1 method for reflecting on learned subjects. After each video you watch, you will be asked to do the following:
\begin{itemize}
    \item Write down 3 things you learned from the video.
    \item Write down 2 things you found interesting from the video.
    \item Write down 1 question you still have after watching the video.
\end{itemize}

You will find the questions and the link to submit your reflection via the link \url{https://partici.fi/52778625}.


\begin{ex}
     Watch the video "But what is a neural network?" by 3Blue1Brown on how a neural network works (You can stop after roughly 16:30 minutes).
     
     Link: \url{https://youtu.be/aircAruvnKk?si=bLMbpxkFR6ZV2ZVr}

     Use the 3-2-1 method to reflect on the video. 

\end{ex}
\begin{ex}
     Watch the video "Gradient descent, how neural networks learn" by 3Blue1Brown (You can stop after roughly 16:38 minutes).
     
     Link: \url{https://youtu.be/IHZwWFHWa-w?si=HYSb311pByrrPoT9}

     Use the 3-2-1 method to provide feedback on the video.

\end{ex}
\begin{ex}
     Watch the video "What is backpropagation really doing?" by 3Blue1Brown (You can stop after roughly 12:30 minutes).
     
     Link: \url{https://youtu.be/Ilg3gGewQ5U?si=qhZGE8Bpjs7wX_kZ}

     Use the 3-2-1 method to provide feedback on the video.

\end{ex}

\begin{ex}
     Go to \url{https://teachablemachine.withgoogle.com/} and try training your own neural network.
\end{ex}


\newpage
\section*{Aufgabe 1: Wie funktioniert ein neuronales Netzwerk?}

\subsection*{Die 3-2-1 Methode}
Für die folgende Aufgabe werden Sie gebeten, die 3-2-1 Methode anzuwenden, um über gelernte Themen nachzudenken. Nachdem Sie jedes Video angeschaut haben, sollen Sie Folgendes tun:
\begin{itemize}
    \item Notieren Sie sich 3 Dinge, die Sie aus dem Video gelernt haben.
    \item Notieren Sie sich 2 interessante Dinge aus dem Video.
    \item Notieren Sie sich 1 Frage, die Sie noch haben, nachdem Sie das Video gesehen haben.
\end{itemize}

Die Fragen und den Link zum Einreichen Ihrer Reflexion finden Sie unter dem Link \url{https://partici.fi/80912744}.


\begin{ex}
     Schauen Sie sich das Video "But what is a neural network?" von 3Blue1Brown über die Funktionsweise eines neuronalen Netzwerks an (Sie können nach ungefähr 16:30 Minuten stoppen). Sie können die deutschen Untertitel einblenden.
     
     Link: \url{https://youtu.be/aircAruvnKk?si=bLMbpxkFR6ZV2ZVr}

     Verwenden Sie die 3-2-1 Methode, um über das Video zu reflektieren. 

\end{ex}
\begin{ex}
     Schauen Sie sich das Video "Gradient descent, how neural networks learn" von 3Blue1Brown an (Sie können nach ungefähr 16:38 Minuten stoppen). Sie können die deutschen Untertitel einblenden.
     
     Link: \url{https://youtu.be/IHZwWFHWa-w?si=HYSb311pByrrPoT9}

     Verwenden Sie die 3-2-1 Methode, um Feedback zum Video zu geben.

\end{ex}
\begin{ex}
     Schauen Sie sich das Video "What is backpropagation really doing?" von 3Blue1Brown an (Sie können nach ungefähr 12:30 Minuten stoppen). Sie können die deutschen Untertitel einblenden.
     
     Link: \url{https://youtu.be/Ilg3gGewQ5U?si=qhZGE8Bpjs7wX_kZ}

     Verwenden Sie die 3-2-1 Methode, um Feedback zum Video zu geben.

\end{ex}

\begin{ex}
     Gehen Sie zu \url{https://teachablemachine.withgoogle.com/} und versuchen Sie, Ihr eigenes neuronales Netzwerk zu trainieren.
\end{ex}


\newpage

\section*{Assignment 2: Using teachablemachine}
Once you have learned about the technical aspoect of neural networks, you will now use the tool teachablemachine to train your own neural network.


\begin{ex}
    Go to \url{https://teachablemachine.withgoogle.com/} and train your own neural network. You can choose between image, sound, and pose training. Play around with the tool and try a few basic examples.

    You can find ideas under the following links:
    \begin{itemize}
     \item \url{https://medium.com/@warronbebster/teachable-machine-tutorial-bananameter-4bfffa765866}
     \item \url{https://medium.com/@warronbebster/teachable-machine-tutorial-head-tilt-f4f6116f491}
     \item \url{https://medium.com/@warronbebster/teachable-machine-tutorial-snap-clap-whistle-4212fd7f3555}
     \item \url{https://vocal.media/education/teachable-machine-project-ideas} 
    \end{itemize}
\end{ex}

\begin{ex}
      In groups of 2 try to create a robust model using teachablemachine. You can choose between image, sound, and pose training. Try to come up with a creative idea for your model.

      When you have finished swap models with another group and try to trick their model. What can you do to make the models fail?
\end{ex}

\begin{ex}
      Watch the video "Teachable Machine 1: Image Classification" (Link: \url{https://www.youtube.com/watch?v=kwcillcWOg0}).
\end{ex}

We will now use the tool teachablemachine to train a neural network and embed it into our own webpage.

\begin{ex}
      Accept the following assignement and clone the project to your personal computer:
      
      \url{https://classroom.github.com/a/ZdMeSBZY}

      You can now edit line 5 of the file \verb|main.js| to include your own model. You can get the link to your model in the teachablemachine tool by clicking on \verb|Export Model|.

     You can now run the sample project using \verb|npm run dev|.

     This project uses the p5.js library to create a canvas and simple drawing tools. You can find the documentation for p5.js under the following link: \url{https://p5js.org/reference/}.
\end{ex}

\begin{ex}
      Use teachablemachine and the sample project to create an AI that can play rock-paper-scissors with you. The machine should always win.
\end{ex}

\newpage

\section*{Aufgabe 2: Verwendung von Teachable Machine}
Nachdem Sie sich mit den technischen Aspekten neuronaler Netzwerke vertraut gemacht haben, werden Sie nun das Tool Teachable Machine verwenden, um Ihr eigenes neuronales Netzwerk zu trainieren.

\begin{ex}
    Gehen Sie zu \url{https://teachablemachine.withgoogle.com/} und trainieren Sie Ihr eigenes neuronales Netzwerk. Sie können zwischen Bild-, Ton- und Pose-Training wählen. Experimentieren Sie mit dem Tool und probieren Sie ein paar grundlegende Beispiele aus.

    Ideen finden Sie unter den folgenden Links:
    \begin{itemize}
        \item \url{https://medium.com/@warronbebster/teachable-machine-tutorial-bananameter-4bfffa765866}
        \item \url{https://medium.com/@warronbebster/teachable-machine-tutorial-head-tilt-f4f6116f491}
        \item \url{https://medium.com/@warronbebster/teachable-machine-tutorial-snap-clap-whistle-4212fd7f3555}
        \item \url{https://vocal.media/education/teachable-machine-project-ideas}
    \end{itemize}
\end{ex}

\begin{ex}
    In Gruppen von 2 Personen versuchen Sie, ein robustes Modell mit Teachable Machine zu erstellen. Sie können zwischen Bild-, Ton- und Pose-Training wählen. Überlegen Sie sich eine kreative Idee für Ihr Modell.

    Wenn Sie fertig sind, tauschen Sie Modelle mit einer anderen Gruppe aus und versuchen Sie, ihr Modell zu täuschen. Was können Sie tun, um die Modelle scheitern zu lassen?
\end{ex}

\begin{ex}
    Schauen Sie sich das Video "Teachable Machine 1: Bildklassifizierung" an (Link: \url{https://www.youtube.com/watch?v=kwcillcWOg0}).
\end{ex}

Wir werden nun das Tool Teachable Machine verwenden, um ein neuronales Netzwerk zu trainieren und in unsere eigene Webseite einzubetten.

\begin{ex}
    Akzeptieren Sie den folgende Auftrag und kopieren Sie das Projekt auf Ihren eigenen Computer:

    \url{https://classroom.github.com/a/l93ysMPK}

    Bearbeiten Sie nun Zeile 5 der Datei \verb|main.js|, um Ihr eigenes Modell einzuschließen. Den Link zu Ihrem Modell erhalten Sie im Teachable Machine Tool, indem Sie auf \verb|Export Model| klicken.

    Sie können das Beispielprojekt jetzt mit \verb|npm run dev| ausführen.

    Dieses Projekt verwendet die p5.js-Bibliothek, um eine Leinwand und einfache Zeichenwerkzeuge zu erstellen. Die Dokumentation für p5.js finden Sie unter folgendem Link: \url{https://p5js.org/reference/}.
\end{ex}

\begin{ex}
    Verwenden Sie Teachable Machine und das Beispielprojekt, um eine KI zu erstellen, die mit Ihnen Schere-Stein-Papier spielen kann. Die Maschine sollte immer gewinnen.
\end{ex}

\newpage

\section*{Assignment 3: AI, Data and Ethics}
During the next few weeks you will work on a short essay on a topic concerning ethics in connection with AI and data. 

\subsubsection*{The Project}
The project should consist of the following parts:
\renewcommand{\labelenumi}{\arabic{enumi}.}

\begin{enumerate}
    \item Write a brief introduction into the training of an AI. This should cover the data selection and the training process. This topic should give a novice reader the necessary background information to understand the following parts of your essay.
    \item Reasearch  a specific topic in the field of AI, data and ethics. 
    \item Collect relevant sources and cite them appropriately.
    \item Summarize your findings and write an essay on the topic. The essay must be clearly recognizable as your own work.
    \item Submit your essay for review by a fellow student and also review a fellow student's essay.
    \item Revise your essay based on the feedback you received.
\end{enumerate}

The draft, your review of another student's essay and the final revised essay will be part of your final grade.

% \subsubsection*{Introduction}
% To give a brief 
% Your essay should be about 1-2 pages long and should include at least 3 primary sources.

\subsubsection*{Writing a Review}
When reviewing a fellow student's essay, you should consider the following points:
\begin{itemize}
    % \item Are the sources cited correctly?
    \item Are the arguments presented in a clear and concise manner?
    \item Are the conclusions drawn from the research valid?
    \item Are there any points that need further clarification?
\end{itemize}
Your review should clearly state for which student you are writing the review. You should highlight the positive aspects of the essay and also point out areas that need improvement. Provide constructive feedback that will help the student to improve their essay.

% Your final grade will be based on the following criteria:
% \begin{itemize}
%     \item Quality of the research
%     \item Quality of the writing
%     \item Quality of the feedback
%     \item Quality of the revision
% \end{itemize}

% You may use the attached list of topic ideas or choose your own topic.

\subsubsection*{Revision}
For the final version of your work, you should incorporate the feedback from the review. At this stage, only minor changes should occur, and the entire work should not be rewritten.

\subsubsection*{Formalities}
Accept the following assignment and clone the project to your personal computer:
\url{https://classroom.github.com/a/QCaKL2Fy}

The essay should be written in markdown and should be submitted as a markdown file. Markdown is a simple markup language that is easy to learn. You can find a guide to markdown here: \url{https://www.markdownguide.org/basic-syntax/}.

The template repository already has the necessary files for you to get started. You will just have to rename them to match your the format required below.


\subsubsection*{Deadlines and Hand In}

Hand in First draft: 21.5.2024 (before the lesson).
The first draft must be handed in as a markdown file with the name \verb|draft-firstname-lastname.md|.

Hand in Review: 28.5.2024 (before the lesson).
The review must be handed in as a markdown file with the name \verb|review-firstname-lastname.md|.

Hand in Final version: 31.5.2024 (23:59).
The final version must be handed in as a markdown file with the name \verb|final-firstname-lastname.md|.


\newpage

\section*{Aufgabe 3: KI, Daten und Ethik}
In den nächsten Wochen werden Sie an einem kurzen Essay zu einem Thema arbeiten, das sich mit Ethik in Verbindung mit KI und Daten befasst.

\subsubsection*{Das Projekt}
Das Projekt sollte aus folgenden Teilen bestehen:

\begin{enumerate}
\item Verfassen Sie eine kurze Einführung in das Training einer KI. Dies sollte die Datenauswahl und den Schulungsprozess abdecken. Dieses Thema sollte einem Anfänger die notwendigen Hintergrundinformationen vermitteln, um die folgenden Teile Ihres Essays zu verstehen.
\item Recherchieren Sie ein spezifisches Thema im Bereich KI, Daten und Ethik.
\item Sammeln Sie relevante Quellen und zitieren Sie diese angemessen.
\item Fassen Sie Ihre Ergebnisse zusammen und verfassen Sie einen Essay zu dem Thema. Der Essay muss klar als Ihr eigenes Werk erkennbar sein.
\item Reichen Sie Ihren Essay zum Review durch eine*n Mitschüler*in ein und überprüfen Sie auch den Essay eines Mitschülers.
\item Überarbeiten Sie Ihren Essay basierend auf dem erhaltenen Feedback.
\end{enumerate}

Der Entwurf, Ihr Review des Essays eines/r Mitschüler/ins und der endgültig überarbeitete Essay werden Teil Ihrer Abschlussnote sein.

\subsubsection*{Das Schreiben einer Überprüfung}
Bei der Überprüfung des Essays eines/r Kolleg/ins sollten Sie die folgenden Punkte berücksichtigen:

\begin{itemize}
\item Werden die Argumente klar und prägnant präsentiert?
\item Sind die Schlussfolgerungen aus der Forschung gültig?
\item Gibt es Punkte, die weiterer Erläuterung bedürfen?
\end{itemize}

Ihre Überprüfung sollte deutlich angeben, für welche*n Kolleg*in Sie die Überprüfung schreiben. Sie sollten die positiven Aspekte des Essays hervorheben und auch Bereiche aufzeigen, die verbessert werden müssen. Geben Sie konstruktives Feedback, das dem/der Schüler/in hilft, seinen/ihren Essay zu verbessern.

\subsubsection*{Überarbeitung}
In der endgültigen Version Ihrer Arbeit sollten Sie das Feedback aus der Überprüfung einarbeiten. Zu diesem Zeitpunkt sollten nur geringfügige Änderungen vorgenommen werden, und die gesamte Arbeit sollte nicht neu geschrieben werden.

\subsubsection*{Formalitäten}
Akzeptieren Sie die folgende Aufgabe und klonen Sie das Projekt auf Ihren Computer:
\url{https://classroom.github.com/a/LPch9Ced}

Der Essay sollte in Markdown verfasst und als Markdown-Datei eingereicht werden. Markdown ist eine einfache Auszeichnungssprache, die leicht zu erlernen ist. Eine Anleitung zur Markdown-Syntax finden Sie hier: \url{https://www.markdownguide.org/basic-syntax/}.

Das Vorlagen-Repository enthält bereits die erforderlichen Dateien, um loszulegen. Sie müssen sie nur umbenennen, um dem unten angegebenen Format zu entsprechen.

\subsubsection*{Fristen und Abgabe}

Einreichung des ersten Entwurfs: 21.5.2024 (vor der Unterrichtsstunde). Der erste Entwurf muss als Markdown-Datei mit dem Namen \verb|draft-vorname-nachname.md| eingereicht werden.

Einreichung der Überprüfung: 28.5.2024 (vor der Unterrichtsstunde). Die Überprüfung muss als Markdown-Datei mit dem Namen \verb|review-vorname-nachname.md| eingereicht werden.

Einreichung der endgültigen Version: 31.5.2024 (23:59 Uhr). Die endgültige Version muss als Markdown-Datei mit dem Namen \verb|final-vorname-nachname.md| eingereicht werden.


\newpage


\section*{Assignment 2a: teachablemachine}

\begin{ex}
    If you have not done so already, watch the video "Teachable Machine 1: Image Classification" (Link: \url{https://www.youtube.com/watch?v=kwcillcWOg0}).
\end{ex}

We will now use the tool teachablemachine to train a neural network and embed it into our own webpage.

\begin{ex}
    Accept the following assignement and clone the project to your personal computer:
    
    \url{https://classroom.github.com/a/ZdMeSBZY}

    You can now edit line 5 of the file \verb|main.js| to include your own model. You can get the link to your model in the teachablemachine tool by clicking on \verb|Export Model|.

   You can now run the sample project using \verb|npm run dev|.

   This project uses the p5.js library to create a canvas and simple drawing tools. You can find the documentation for p5.js under the following link: \url{https://p5js.org/reference/}.
\end{ex}

\begin{ex}
    Use teachablemachine and the sample project to create an AI that can play rock-paper-scissors with you. The machine should always win.
\end{ex}

\begin{ex}
     Watch the 2nd video on teachablemachine: \url{https://www.youtube.com/watch?v=UPgxnGC8oBU}
\end{ex}


\begin{ex}
     Create your own game project using teachablemachine.
\end{ex}



\section*{Artificial Intelligence and Ethics}





\section*{Topics}
\emph{Main source: \url{https://www.d.umn.edu/~tcolburn/cs3111/presentation/presentation.xhtml}}
\subsection*{Networked Communications}
\begin{itemize}
\item Email (spam/phishing)
\item Internet censorship and freedom of expression
\item Identity theft
\item Cyberbullying
\item Internet addiction
\item Social media immersion
\item Ad Blocking on Websites
\item Net neutrality
\item Children, predators, and inappropriate content
\item Human trafficking and the internet
\item Misogyny and the gaming community
\item Gaming and violence
\item Massively multiplayer online role-playing games
\item Internet of things
\item Digital advertising
\item Crowdfunding
\item Filter/Content Bubbles in Networks
\item Fake News
\item Online Education
\end{itemize}
\subsection*{Intellectual Property} 
\begin{itemize}
\item Restrictions on fair use (DMCA, DRM)
\item Digital media distribution
\item Peer-to-peer networks
\item Pirate Bay
\item Piracy and video games
\item Piracy and music
\item Hardware modding and jailbreaking
\item Software copyright and patents
\item Software licensing
\item Open-source software (OSS)
\item The FSF and software freedom
\item Creative Commons
\item Intellectual Property on the Cloud
\end{itemize}

\subsection*{Information Privacy}
\begin{itemize}
\item "Big Data" and data mining
\item Medical data and patient privacy
\item Covert surveillance and wiretapping
\item NSA
\item Encryption
\item Edward Snowden
\item The Patriot Act
\item Drones
\item The Tor Browser
\item Cyberstalking
\item Nanny Surveillance
\item Doxxing
\end{itemize}

\subsection*{Computer and Network Security}
\begin{itemize}
\item Massive data breaches
\item Ethical hacking
\item Social engineering
\item Hacktivism
\item DDoS Attacks
\item Malware (spam, viruses, worms, rootkits)
\item Online voting
\item Cyber crime
\item Cyber warfare
\item OSS and security
\item Wikileaks
\item Medical Devices
\end{itemize}

\subsection*{Computer Reliability}
\begin{itemize}
\item Software issues (errors, failures, warranties)
\item Software verification and validation
\item Computer simulations
\item Volkswagen software scandal
\item Planned Obsolesence
\end{itemize}

\subsection*{Work, Wealth, and Society}
\begin{itemize}
\item Automation and unemployment
\item Effects of globalization
\item Digital divide
\item Cryptocurrency
\item Blockchain and its Impact
\item Gender diversity in CS/STEM
\item Virtual reality
\item Whistleblowing
\end{itemize}

\subsection*{Artificial Intelligence and Artificial Life}
\begin{itemize}
\item "Racist" and "Sexist" AI
\item Autonomous vehicles
\item Turing test
\item Electronic singularity
\item Intelligence amplification and cyborgs
\item Transhumanism
\item Nanotechnology
\item Creating moral robots
\item Rights of robots
\item Robotic surgery
\item Automating Finance
\end{itemize}

\end{document}
