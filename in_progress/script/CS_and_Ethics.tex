\documentclass[12pt,a4paper]{report}

\usepackage{xcolor}
\def\farbe{blue}

\usepackage{dcsls}

\usepackage{circuitikz}

\ctikzset{
    logic ports=ieee,
    logic ports/scale=0.8,
    logic ports/fill=lightgray
}

\usetikzlibrary{arrows,shapes.gates.logic.US,shapes.gates.logic.IEC,calc}


\usepackage{listings}
\lstset{language=Python}


\definecolor{codegreen}{rgb}{0,0.6,0}
\definecolor{codegray}{rgb}{0.9,0.9,0.9}
\definecolor{codepurple}{rgb}{0.58,0,0.82}
\definecolor{backcolour}{rgb}{0.95,0.95,0.92}

\lstdefinestyle{mystyle}{
%	morekeywords={forward,turn},
    backgroundcolor=\color{codegray},
    commentstyle=\color{codegreen},
%    keywordstyle=\color{codegreen},
    numberstyle=\tiny\color{gray},
    stringstyle=\color{codepurple},
    basicstyle=\footnotesize,
    identifierstyle=\color{blue},
    stringstyle=\color{orange},
    breakatwhitespace=false,
    breaklines=true,
    captionpos=b,
    keepspaces=true,
    numbers=left,
    numbersep=5pt,
    showspaces=false,
    showstringspaces=false,
    showtabs=false,
    tabsize=2
}

\lstset{style=mystyle}

\setlength{\headheight}{25pt}


%%% Fancy Header and Footer
\renewcommand{\headrule}{\vbox to 0pt{\hbox to\headwidth{\color{\farbe}\rule{\headwidth}{1pt}}\vss}}
\pagestyle{fancy} %eigener Seitenstil
\fancyhf{} %alle Kopf- und Fusszeilenfelder bereinigen
\fancyhead[C]{Artificial Intelligence and Ethics} %Kopfzeile mitte
%\fancyhead[R]{\includegraphics[width=0.2cm]{x.png}}
\fancyfoot[C]{\thepage}


\newcommand{\bfb}[1]{{\bf \color{blue} #1}}




\begin{document}

\section*{Assignment 1: How does a neural network work?}

\subsection*{The 3-2-1 Method}
For the following assignment you will be asked to use the 3-2-1 method for reflecting on learned subjects. After each video you watch, you will be asked to do the following:
\begin{itemize}
    \item Write down 3 things you learned from the video.
    \item Write down 2 things you found interesting from the video.
    \item Write down 1 question you still have after watching the video.
\end{itemize}

You will find the questions and the link to submit your reflection via the link \url{https://partici.fi/52778625}.


\begin{ex}
     Watch the video "But what is a neural network?" by 3Blue1Brown on how a neural network works (You can stop after roughly 16:30 minutes).
     
     Link: \url{https://youtu.be/aircAruvnKk?si=bLMbpxkFR6ZV2ZVr}

     Use the 3-2-1 method to reflect on the video. 

\end{ex}
\begin{ex}
     Watch the video "Gradient descent, how neural networks learn" by 3Blue1Brown (You can stop after roughly 16:38 minutes).
     
     Link: \url{https://youtu.be/IHZwWFHWa-w?si=HYSb311pByrrPoT9}

     Use the 3-2-1 method to provide feedback on the video.

\end{ex}
\begin{ex}
     Watch the video "What is backpropagation really doing?" by 3Blue1Brown (You can stop after roughly 12:30 minutes).
     
     Link: \url{https://youtu.be/Ilg3gGewQ5U?si=qhZGE8Bpjs7wX_kZ}

     Use the 3-2-1 method to provide feedback on the video.

\end{ex}

\begin{ex}
     Go to \url{https://teachablemachine.withgoogle.com/} and try training your own neural network.
\end{ex}


\newpage
\section*{Aufgabe 1: Wie funktioniert ein neuronales Netzwerk?}

\subsection*{Die 3-2-1 Methode}
Für die folgende Aufgabe werden Sie gebeten, die 3-2-1 Methode anzuwenden, um über gelernte Themen nachzudenken. Nachdem Sie jedes Video angeschaut haben, sollen Sie Folgendes tun:
\begin{itemize}
    \item Notieren Sie sich 3 Dinge, die Sie aus dem Video gelernt haben.
    \item Notieren Sie sich 2 interessante Dinge aus dem Video.
    \item Notieren Sie sich 1 Frage, die Sie noch haben, nachdem Sie das Video gesehen haben.
\end{itemize}

Die Fragen und den Link zum Einreichen Ihrer Reflexion finden Sie unter dem Link \url{https://partici.fi/80912744}.


\begin{ex}
     Schauen Sie sich das Video "But what is a neural network?" von 3Blue1Brown über die Funktionsweise eines neuronalen Netzwerks an (Sie können nach ungefähr 16:30 Minuten stoppen). Sie können die deutschen Untertitel einblenden.
     
     Link: \url{https://youtu.be/aircAruvnKk?si=bLMbpxkFR6ZV2ZVr}

     Verwenden Sie die 3-2-1 Methode, um über das Video zu reflektieren. 

\end{ex}
\begin{ex}
     Schauen Sie sich das Video "Gradient descent, how neural networks learn" von 3Blue1Brown an (Sie können nach ungefähr 16:38 Minuten stoppen). Sie können die deutschen Untertitel einblenden.
     
     Link: \url{https://youtu.be/IHZwWFHWa-w?si=HYSb311pByrrPoT9}

     Verwenden Sie die 3-2-1 Methode, um Feedback zum Video zu geben.

\end{ex}
\begin{ex}
     Schauen Sie sich das Video "What is backpropagation really doing?" von 3Blue1Brown an (Sie können nach ungefähr 12:30 Minuten stoppen). Sie können die deutschen Untertitel einblenden.
     
     Link: \url{https://youtu.be/Ilg3gGewQ5U?si=qhZGE8Bpjs7wX_kZ}

     Verwenden Sie die 3-2-1 Methode, um Feedback zum Video zu geben.

\end{ex}

\begin{ex}
     Gehen Sie zu \url{https://teachablemachine.withgoogle.com/} und versuchen Sie, Ihr eigenes neuronales Netzwerk zu trainieren.
\end{ex}


\newpage


\section*{Artificial Intelligence and Ethics}





\section*{Topics}
\emph{Main source: \url{https://www.d.umn.edu/~tcolburn/cs3111/presentation/presentation.xhtml}}
\subsection*{Networked Communications}
\begin{itemize}
\item Email (spam/phishing)
\item Internet censorship and freedom of expression
\item Identity theft
\item Cyberbullying
\item Internet addiction
\item Social media immersion
\item Ad Blocking on Websites
\item Net neutrality
\item Children, predators, and inappropriate content
\item Human trafficking and the internet
\item Misogyny and the gaming community
\item Gaming and violence
\item Massively multiplayer online role-playing games
\item Internet of things
\item Digital advertising
\item Crowdfunding
\item Filter/Content Bubbles in Networks
\item Fake News
\item Online Education
\end{itemize}
\subsection*{Intellectual Property} 
\begin{itemize}
\item Restrictions on fair use (DMCA, DRM)
\item Digital media distribution
\item Peer-to-peer networks
\item Pirate Bay
\item Piracy and video games
\item Piracy and music
\item Hardware modding and jailbreaking
\item Software copyright and patents
\item Software licensing
\item Open-source software (OSS)
\item The FSF and software freedom
\item Creative Commons
\item Intellectual Property on the Cloud
\end{itemize}

\subsection*{Information Privacy}
\begin{itemize}
\item "Big Data" and data mining
\item Medical data and patient privacy
\item Covert surveillance and wiretapping
\item NSA
\item Encryption
\item Edward Snowden
\item The Patriot Act
\item Drones
\item The Tor Browser
\item Cyberstalking
\item Nanny Surveillance
\item Doxxing
\end{itemize}

\subsection*{Computer and Network Security}
\begin{itemize}
\item Massive data breaches
\item Ethical hacking
\item Social engineering
\item Hacktivism
\item DDoS Attacks
\item Malware (spam, viruses, worms, rootkits)
\item Online voting
\item Cyber crime
\item Cyber warfare
\item OSS and security
\item Wikileaks
\item Medical Devices
\end{itemize}

\subsection*{Computer Reliability}
\begin{itemize}
\item Software issues (errors, failures, warranties)
\item Software verification and validation
\item Computer simulations
\item Volkswagen software scandal
\item Planned Obsolesence
\end{itemize}

\subsection*{Work, Wealth, and Society}
\begin{itemize}
\item Automation and unemployment
\item Effects of globalization
\item Digital divide
\item Cryptocurrency
\item Blockchain and its Impact
\item Gender diversity in CS/STEM
\item Virtual reality
\item Whistleblowing
\end{itemize}

\subsection*{Artificial Intelligence and Artificial Life}
\begin{itemize}
\item "Racist" and "Sexist" AI
\item Autonomous vehicles
\item Turing test
\item Electronic singularity
\item Intelligence amplification and cyborgs
\item Transhumanism
\item Nanotechnology
\item Creating moral robots
\item Rights of robots
\item Robotic surgery
\item Automating Finance
\end{itemize}

\end{document}
