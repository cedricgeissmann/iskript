\documentclass[11pt,a4paper]{report}

\usepackage{xcolor}
\def\farbe{blue}

\usepackage{dclecture}

\usepackage{circuitikz}

\ctikzset{
    logic ports=ieee,
    logic ports/scale=0.8,
    logic ports/fill=lightgray
}

\usetikzlibrary{arrows,shapes.gates.logic.US,shapes.gates.logic.IEC,calc}


\usepackage{listings}
\lstset{language=Python}

\usepackage{longtable}

\usepackage{verbatim} % Added for the folder structure example

\definecolor{codegreen}{rgb}{0,0.6,0}
\definecolor{codegray}{rgb}{0.9,0.9,0.9}
\definecolor{codepurple}{rgb}{0.58,0,0.82}
\definecolor{backcolour}{rgb}{0.95,0.95,0.92}

\lstdefinestyle{mystyle}{
%	morekeywords={forward,turn},
    backgroundcolor=\color{codegray},
    commentstyle=\color{codegreen},
%    keywordstyle=\color{codegreen},
    numberstyle=\tiny\color{gray},
    stringstyle=\color{codepurple},
    basicstyle=\footnotesize,
    identifierstyle=\color{blue},
    stringstyle=\color{orange},
    breakatwhitespace=false,
    breaklines=true,
    captionpos=b,
    keepspaces=true,
    numbers=left,
    numbersep=5pt,
    showspaces=false,
    showstringspaces=false,
    showtabs=false,
    tabsize=2
}

\lstset{style=mystyle}




%%% Fancy Header and Footer
\renewcommand{\headrule}{\vbox to 0pt{\hbox to\headwidth{\color{\farbe}\rule{\headwidth}{1pt}}\vss}}
\pagestyle{fancy} %eigener Seitenstil
\fancyhf{} %alle Kopf- und Fusszeilenfelder bereinigen
\fancyhead[C]{Computer Science} %Kopfzeile mitte
%\fancyhead[R]{\includegraphics[width=0.2cm]{x.png}}
\fancyfoot[C]{\thepage}


\newcommand{\bfb}[1]{{\bf \color{blue} #1}}




\begin{document}
\section{Computer Basics}

\subsection{Understanding the File System}

Think of your computer's file system as a digital filing cabinet. Instead of drawers and paper folders, you have drives and digital folders to organize your documents. The main tool to navigate this system is the \textbf{File Explorer} (on Windows) or \textbf{Finder} (on macOS).

Use \verb|win-e| (Windows) or \verb|cmd-n| (from the Desktop on Mac) to open a new window.


These windows show the contents of your computer.
Your main folder is called the \textbf{Home} folder, and it's personalized for your user account. It contains several key subfolders:

\begin{itemize}
    \item \textbf{Desktop:} Files you place here are visible on your main screen background. Good for temporary files, but can get cluttered quickly!
    \item \textbf{Documents:} The intended place for most of your personal work, like essays, spreadsheets, and reports.
    \item \textbf{Downloads:} Your web browser's default location for saving files from the internet. It's a good idea to move important files out of here and into \emph{Documents}.
    \item \textbf{Pictures, Music, Videos:} Folders designated for specific types of media.
\end{itemize}

You will also likely see cloud storage folders like \textbf{OneDrive}, \textbf{Google Drive}, or \textbf{Dropbox} integrated into your file browser.


\subsection{Files, Folders, and Extensions: Your Digital Building Blocks}

On your phone, apps manage files for you, so you rarely see the underlying structure. On a computer, understanding this structure is key to staying organized.

\paragraph{Folders (or Directories)} are containers. This used to refer to actual folders in a filing cabinet. You use them to group related files together. A folder can hold both files and other folders (called subfolders).

\paragraph{Files} are the actual documents themselves—the essay, the photo, the spreadsheet. Every file has two parts: a \textbf{name} and an \textbf{extension}, separated by a dot. For example: \texttt{Lab-Report.pdf}.

\paragraph{File Extensions} tell the operating system what kind of file it is and what program should be used to open it. For example, your computer knows to open a \texttt{.docx} file with Microsoft Word and a \texttt{.jpg} file with a photo viewer.

Here are some common extensions you'll encounter:
\begin{center}
\begin{tabular}{|l|p{0.6\textwidth}|}
\hline
\bfb{Extension} & \bfb{Description} \\
\hline
\texttt{.docx, .pptx, .xlsx} & Microsoft Office files (Word, PowerPoint, Excel). \\
\texttt{.pdf} & Portable Document Format. Great for sharing final versions of documents because they are hard to edit and look the same everywhere. \\
\texttt{.jpg, .png, .gif} & Standard image formats for photos and graphics. \\
\texttt{.mp3, .mp4, .mov} & Audio and video files. \\
\texttt{.zip} & A compressed archive. It's a single file that bundles other files and folders together to make them smaller and easier to transfer. \\
\texttt{.txt} & A plain text file with no formatting. \\
\texttt{.py, .java, .m} & Code or script files for programming languages like Python, Java, or MATLAB. \\
\hline
\end{tabular}
\end{center}

\paragraph{Building a Good Folder Structure}
The best way to stay organized is to create a logical hierarchy of folders. Start broad and get more specific. We recommend organizing by semester, then by course.

\newpage
Here is an example structure inside your main \texttt{Documents} folder:
\begin{verbatim}
My-Studies/
|
+--- Fall-2025/
|    |
|    +--- CS101_Intro-to-Programming/
|    |    |
|    |    +--- Homework/
|    |    |    '--- 2025-09-15_Homework-1.py
|    |    |
|    |    +--- Lectures/
|    |    |    '--- 2025-09-08_Lecture-Notes.pdf
|    |    |
|    |    '--- Project/
|    |
|    '--- MATH150_Calculus-I/
|         |
|         +--- Notes/
|         |
|         '--- Worksheets/
|
'--- Spring-2026/
     |
     '--- (etc...)
\end{verbatim}

By combining a good folder structure with the smart naming conventions discussed next, you will always be able to find your work quickly.

\subsection{Local vs. Cloud Storage}

You can store files either \textbf{locally} (directly on your computer's hard drive) or on the \textbf{cloud} (on a remote server accessed via the internet). Each has pros and cons.

\begin{center}
\begin{tabular}{|l|p{0.4\textwidth}|p{0.4\textwidth}|}
\hline
 & \bfb{Local Storage} & \bfb{Cloud Storage} \\
\hline
\bf{Pros} & Fast access, no internet required, full privacy control. & Access from any device, easy to share and collaborate, syncs automatically. \\
\hline
\bf{Cons} & Hard to access from other devices, vulnerable to hardware failure (e.g. laptop breaks). & Requires internet, potential privacy concerns, may have storage limits or costs. \\
\hline
\end{tabular}
\end{center}

You need to figure out a good system for managing documents that works for you so you can always find what you need.
For most people, a \textbf{hybrid approach} is best:
\begin{itemize}
    \item Use a cloud service (like OneDrive, provided by the school) for your active, important documents (school work, projects). This keeps them synced and accessible.
    \item Work directly from the synchronized cloud folder on your computer.
        Most modern cloud services handle this seamlessly.
    \item Use your local \emph{Downloads} or \emph{Desktop} folders for temporary, less important files.
\end{itemize}


\subsection{Smart Naming Conventions}
How you name your files and folders matters! A good system makes sorting and finding files much easier.

\paragraph{The Rules:}
\begin{enumerate}
    \item \textbf{Avoid special characters:} Stick to letters, numbers, hyphens (-), and underscores (\_). Avoid characters like \verb|? / \ : * " < > | |.
    \item \textbf{Avoid spaces:} While modern systems can handle spaces, they can cause problems with programming scripts or when transferring files.
        Use \texttt{kebab-case} (my-file-name), \texttt{snake\_case} (my\_file\_name), or \texttt{CamelCase} (MyFileName) instead.
    \item \textbf{Be descriptive but brief:} \texttt{Biology-Lab-Report-Photosynthesis.docx} is much better than \texttt{Document1.docx}.
    \item \textbf{Use dates for chronological files:} For recurring files like notes, start the filename with the date in \texttt{YYYY-MM-DD} format. This makes sorting by date automatic and reliable.
        
\end{enumerate}

\begin{center}
\begin{tabular}{|l|l|}
\hline
\bfb{Good Example} & \bfb{Bad Example} \\
\hline
\texttt{2025-08-11\_CS-Lecture-Notes.pdf} & \texttt{notes 11.08.pdf} \\
\texttt{Physics-Homework-Week3.docx} & \texttt{Phys HW \#3.docx} \\
\texttt{Project-Proposal-Draft-v2.pdf} & \texttt{draft for project new version.pdf} \\
\hline
\end{tabular}
\end{center}


\subsection{Backups: Don't Lose Your Work!}

A common mistake is thinking that cloud storage (like OneDrive) is a backup. It's not! Cloud storage is for \textbf{synchronization}, not protection. If your local file gets accidentally deleted or corrupted (damaged and unreadable), the cloud version will be deleted or corrupted too.


A true backup is a separate copy of your files stored in a different location. The gold standard is the \textbf{3-2-1 Backup Rule}:
\begin{itemize}
    \item Have at least \textbf{3} total copies of your data.
    \item Store the copies on \textbf{2} different types of media (e.g., your laptop's internal drive and an external hard drive).
    \item Keep \textbf{1} copy off-site (in a different physical location).
        
\end{itemize}

\paragraph{Practical Steps:}
\begin{enumerate}
    \item Get an external hard drive. 
    \item Use your operating system's built-in backup software. It's easy and automatic.
    \begin{itemize}
        \item \textbf{macOS:} Use \emph{Time Machine}.
        \item \textbf{Windows:} Use \emph{File History}.
    \end{itemize}
    \item Run a backup regularly (at least once a week).
\end{enumerate}


\section{Essential Keyboard Shortcuts}
Using shortcuts will dramatically speed up your workflow.

\begin{longtable}{|p{0.3\textwidth}|p{0.3\textwidth}|p{0.3\textwidth}|}
\hline
{\bf Action} & {\bf Windows Shortcut} & {\bf macOS Shortcut}  \\
\hline
\endhead
\multicolumn{3}{|c|}{\bfb{File \& Application Management}} \\
\hline
New Explorer/Finder Window & \verb|Win + E| & \verb|Cmd + N| (on Desktop) \\

New Folder & \verb|Ctrl + Shift + N| & \verb|Cmd + Shift + N| \\
Open New Document/Tab & \verb|Ctrl + N| & \verb|Cmd + N| \\
Save Current File & \verb|Ctrl + S| & \verb|Cmd + S| \\
Print & \verb|Ctrl + P| & \verb|Cmd + P| \\

Close Window/Tab & \verb|Ctrl + W| & \verb|Cmd + W| \\

Close Application & \verb|Alt + F4| & \verb|Cmd + Q| \\
Rename File & \verb|F2| & \verb|Return| \\

\hline
\multicolumn{3}{|c|}{\bfb{Text Editing}} \\
\hline
Undo & \verb|Ctrl + Z| & \verb|Cmd + Z| \\

Redo & \verb|Ctrl + Y| & \verb|Cmd + Shift + Z| \\
Cut & \verb|Ctrl + X| & \verb|Cmd + X| \\

Copy & \verb|Ctrl + C| & \verb|Cmd + C| \\
Paste & \verb|Ctrl + V| & \verb|Cmd + V| \\
Select All & \verb|Ctrl + A| & \verb|Cmd + A| \\

\hline
\multicolumn{3}{|c|}{\bfb{Navigation \& Other Tools}} \\
\hline
Cycle Through Apps & \verb|Alt + Tab| & \verb|Cmd + Tab| \\

Search (Files/Apps) & \verb|Win + S| & \verb|Cmd + Space| \\
Screenshot (Area) & \verb|Win + Shift + S| & \verb|Cmd + Shift + 4| \\
Screenshot (Full Screen) & \verb|Win + PrtScn| & \verb|Cmd + Shift + 3| \\
\hline

\end{longtable}
\end{document}