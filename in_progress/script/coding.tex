\documentclass[english,11pt,a4paper]{report}

\usepackage{xcolor}
\def\farbe{blue}

\usepackage{dclecture}

\usepackage{circuitikz}

\ctikzset{
    logic ports=ieee,
    logic ports/scale=0.8,
    logic ports/fill=lightgray
}

\usetikzlibrary{arrows,shapes.gates.logic.US,shapes.gates.logic.IEC,calc}

\makeatletter

\usepackage{listings}

\definecolor{lightgray}{rgb}{0.95, 0.95, 0.95}
\definecolor{darkgray}{rgb}{0.4, 0.4, 0.4}
%\definecolor{purple}{rgb}{0.65, 0.12, 0.82}
\definecolor{editorGray}{rgb}{0.95, 0.95, 0.95}
\definecolor{editorOcher}{rgb}{1, 0.5, 0} % #FF7F00 -> rgb(239, 169, 0)
\definecolor{editorGreen}{rgb}{0, 0.5, 0} % #007C00 -> rgb(0, 124, 0)
\definecolor{orange}{rgb}{1,0.45,0.13}		
\definecolor{olive}{rgb}{0.17,0.59,0.20}
\definecolor{brown}{rgb}{0.69,0.31,0.31}
\definecolor{purple}{rgb}{0.38,0.18,0.81}
\definecolor{lightblue}{rgb}{0.1,0.57,0.7}
\definecolor{lightred}{rgb}{1,0.4,0.5}
\usepackage{upquote}
\usepackage{listings}
% CSS
\lstdefinelanguage{CSS}{
  keywords={color,background-image:,margin,padding,font,weight,display,position,top,left,right,bottom,list,style,border,size,white,space,min,width, transition:, transform:, transition-property, transition-duration, transition-timing-function},	
  sensitive=true,
  morecomment=[l]{//},
  morecomment=[s]{/*}{*/},
  morestring=[b]',
  morestring=[b]",
  alsoletter={:},
  alsodigit={-}
}

% JavaScript
\lstdefinelanguage{JavaScript}{
  morekeywords={typeof, new, true, false, catch, function, return, null, catch, switch, var, if, in, while, do, else, case, break},
  morecomment=[s]{/*}{*/},
  morecomment=[l]//,
  morestring=[b]",
  morestring=[b]'
}

\lstdefinelanguage{HTML5}{
  language=html,
  sensitive=true,	
  alsoletter={<>=-},	
  morecomment=[s]{<!-}{-->},
  tag=[s],
  otherkeywords={
  % General
  >,
  % Standard tags
	<!DOCTYPE,
  </html, <html, <head, <title, </title, <style, </style, <link, </head, <meta, />,
	% body
	</body, <body,
	% Divs
	</div, <div, </div>, 
	% Paragraphs
	</p, <p, </p>,
	% scripts
	</script, <script,
  % More tags...
  <canvas, /canvas>, <svg, <rect, <animateTransform, </rect>, </svg>, <video, <source, <iframe, </iframe>, </video>, <image, </image>, <header, </header, <article, </article
  },
  ndkeywords={
  % General
  =,
  % HTML attributes
  charset=, src=, id=, width=, height=, style=, type=, rel=, href=,
  % SVG attributes
  fill=, attributeName=, begin=, dur=, from=, to=, poster=, controls=, x=, y=, repeatCount=, xlink:href=,
  % properties
  margin:, padding:, background-image:, border:, top:, left:, position:, width:, height:, margin-top:, margin-bottom:, font-size:, line-height:,
	% CSS3 properties
  transform:, -moz-transform:, -webkit-transform:,
  animation:, -webkit-animation:,
  transition:,  transition-duration:, transition-property:, transition-timing-function:,
  }
}

\lstdefinestyle{htmlcssjs} {%
  % General design
%  backgroundcolor=\color{editorGray},
  basicstyle={\footnotesize\ttfamily},   
  frame=b,
  % line-numbers
  xleftmargin={0.75cm},
  numbers=left,
  stepnumber=1,
  firstnumber=1,
  numberfirstline=true,	
  % Code design
  identifierstyle=\color{black},
  keywordstyle=\color{blue}\bfseries,
  ndkeywordstyle=\color{editorGreen}\bfseries,
  stringstyle=\color{editorOcher}\ttfamily,
  commentstyle=\color{brown}\ttfamily,
  % Code
  language=HTML5,
  alsolanguage=JavaScript,
  alsodigit={.:;},	
  tabsize=2,
  showtabs=false,
  showspaces=false,
  showstringspaces=false,
  extendedchars=true,
  breaklines=true,
  % German umlauts
  literate=%
  {Ö}{{\"O}}1
  {Ä}{{\"A}}1
  {Ü}{{\"U}}1
  {ß}{{\ss}}1
  {ü}{{\"u}}1
  {ä}{{\"a}}1
  {ö}{{\"o}}1
}
%
\lstdefinestyle{py} {%
language=python,
literate=%
*{0}{{{\color{lightred}0}}}1
{1}{{{\color{lightred}1}}}1
{2}{{{\color{lightred}2}}}1
{3}{{{\color{lightred}3}}}1
{4}{{{\color{lightred}4}}}1
{5}{{{\color{lightred}5}}}1
{6}{{{\color{lightred}6}}}1
{7}{{{\color{lightred}7}}}1
{8}{{{\color{lightred}8}}}1
{9}{{{\color{lightred}9}}}1,
basicstyle=\footnotesize\ttfamily, % Standardschrift
numbers=left,               % Ort der Zeilennummern
%numberstyle=\tiny,          % Stil der Zeilennummern
%stepnumber=2,               % Abstand zwischen den Zeilennummern
numbersep=5pt,              % Abstand der Nummern zum Text
tabsize=4,                  % Groesse von Tabs
extendedchars=true,         %
breaklines=true,            % Zeilen werden Umgebrochen
keywordstyle=\color{blue}\bfseries,
frame=b,
commentstyle=\color{brown}\itshape,
stringstyle=\color{editorOcher}\ttfamily, % Farbe der String
showspaces=false,           % Leerzeichen anzeigen ?
showtabs=false,             % Tabs anzeigen ?
xleftmargin=17pt,
framexleftmargin=17pt,
framexrightmargin=5pt,
framexbottommargin=4pt,
%backgroundcolor=\color{lightgray},
showstringspaces=false,      % Leerzeichen in Strings anzeigen ?
}%
%
\makeatother



%%% Fancy Header and Footer
\renewcommand{\headrule}{\vbox to 0pt{\hbox to\headwidth{\color{\farbe}\rule{\headwidth}{1pt}}\vss}}
\setlength{\headheight}{24pt}
\pagestyle{fancy} %eigener Seitenstil
\fancyhf{} %alle Kopf- und Fusszeilenfelder bereinigen
\fancyhead[C]{Computer Science} %Kopfzeile mitte
%\fancyhead[R]{\includegraphics[width=0.2cm]{x.png}}
\fancyfoot[C]{\thepage}


\newcommand{\bfb}[1]{{\bf \color{blue} #1}}




\begin{document}
\section{Some Remarks on Coding}
\bfb{Coding} is a sort of blanket term for any written instructions that get interpreted or executed by a digital device. Coding can be writing HTML for a website, writing an app for your phone or even just programming your smart fridge to remind you to buy orange juice.

Simply put, computer coding is the process of using a programming language to deliver instructions to a computer. The code tells the machine what tasks to perform and how to perform them. These detailed instructions are written in multiple lines of code, and a document full of code is called a script.

The script directs the computer to carry out your desired actions. You have to ensure your code is correct otherwise the computer will not understand your commands. Every script is designed for a specific purpose. Anything from resizing an image to playing a particular sound or video.

Any website you visit, application you use, or piece of technology you interact with, works by following code. In fact, even when you hit ‘Like’ on someone’s social media post, a script is triggered and shares your action with the world\footnote{https://www.thinkful.com/blog/coding-best-practices/}.

The only things you really need to code are a text editor and your head. There are visual coding interfaces such as scratch, but they have the same functionalities as a text editor.

\subsection{Which Text Editor?}

Asking a coder which editor they use can be the start of a massive flame war -- tread carefully.  We have decided to use \bfb{Visual Studio Code} (VS Code) for its mix of ease of use and functionality. 

\subsection{Which Head?}

If you can use anything other than your own head then please see a doctor immediately. Remember: its ok to talk to your computer, but if your computer answers you should seek help.

\subsection{Installing VS Code}
Click on the following link and download and run the installer: 

\url{https://code.visualstudio.com/}

\subsection{Installing Git}
Git is a free and open source version control system. A version control system allows you to track your changes in a project and undo mistakes that you may have built in along the way. 

Click on the following link to install git for your system. If you are running MacOS, git should already be installed (To check this open the 'Terminal' application and type \verb|git -v|).

\url{https://git-scm.com/}

\subsection{Make a Github Account}
Github is a free online code repository (think online folder) that integrates with git and visual studio code. Create a free account here: \url{https://github.com/}

When you open a file in Visual Studio Code you will be able to save it to an online repository. This will also allow us to easily give you basic project templates or other bits of code.

\newpage
\section{Coding Best Practices}

Here we will look at some general best practices for coding. Language specific best practices will be shown later.

Here are four main things to be aware of and do when learning to code: 

\begin{itemize}
\item \bfb{Indentation and readability}: properly indented code is much easier to read than non-indented code. -- it is especially useful for finding where code chunks begin and end. Compare the two code chunks below and you will see that the first one is much easier to understand. Luckily most code editors automatically indent the code as you type.  In VS Code you can highlight a code chunk and press the tab key to indent it by four spaces. Press shift-tab to decrease the indent.

Avoid long lines. It is easier for humans to read blocks of lines that are horizontally short and vertically long. 

Keep code as short as possible without losing readability. A single function should do one task.

\newpage
\begin{multicols}{2}
\begin{lstlisting}[style=htmlcssjs]
<!DOCTYPE html>
<html>
  <head>
    <meta charset="utf8">
    <title>My Title</title>
  </head>
  <body>
      Main body here
  </body>
</html>
\end{lstlisting}

\begin{lstlisting}[style=htmlcssjs]
<!DOCTYPE html>
<html>
<head>
<meta charset="utf8">
<title>My Title</title>
</head>
<body>
Main body here
</body>
</html>
\end{lstlisting}
\end{multicols}

\item \bfb{Meaningful naming}: Give your files, variables, classes etc intuitive names.: e.g.  name a function \verb|findMax()| instead of \verb|f2()|. The name should describe what the variable or function does or will be used for. There are different ways for writing multi-word names depending on the language you use. These will be mentioned in the corresponding sections.

\item \bfb{Dont' repeat yourself}: Also known as the DRY principle, “Don’t repeat yourself” strives to reduce code duplication. The idea here is that if you have code that’s doing the same thing twice, it should be made into a function. By abstracting code into functions, you can reuse that code and make development more efficient. In addition, avoiding code duplication makes debugging easier, as you won’t have to fix a bug in every instance of repeated code throughout your program\footnote{https://www.educative.io/blog/coding-best-practices}.

\item \bfb{Commenting}: Code is for the compiler, while comments are for coders.  In a perfect world good code would be self-explanatory, however in real life this is not always the case.  Your comments should give context and additional explanations. Do not simply describe what the code does.

\item \bfb{Organizing Files}: You should create a new folder for each project. If you use multiple additional resources or images add a subfolder \verb|res| or \verb|img|.  Use lower case for naming files and don't use spaces. 

\end{itemize}

\subsection{HTML Best Practices}
\begin{itemize}
\item \bfb{Starting Page}: Your starting page should always be named index.html
\item \bfb{Use proper document structure\footnote{https://blog.tbhcreative.com/2015/08/10-best-practices-in-html.html}}: HTML documents will still work without elements such as \verb|<html>|, \verb|<head>|, and \verb|<body>|. However, the pages will not render correctly in every browser so it's important to be consistent using the proper document structure. 

\item \bfb{Declare the correct doctype\footnote{https://www.w3schools.com/html/html5\_syntax.asp}}: When creating an HTML document, the first thing to declare is the doctype. This will tell the browser the standards you are using to render your markup correctly. 

\item \bfb{Meta Data}: To ensure proper interpretation and correct search engine indexing, both the language and the character encoding \verb|<meta charset="charset">| should be defined as early as possible in an HTML document.

\item \bfb{Use lowercase element names}: Your HTML code can be written in lowercase or uppercase and the web page will render correctly. However, it is best practice to keep tag names in lowercase because it is easier to read and maintain.

\begin{lstlisting}[style=htmlcssjs]
<!DOCTYPE html>
<html>
  <head>
    <meta charset="UTF-8">
    <title>My Title</title>
  </head>
  <body>
      Main body here
  </body>
</html>
\end{lstlisting}

\item \bfb{Close all HTML tags}: In HTML, you do not have to close all elements (for example the \verb|<p>| element).

However, we strongly recommend closing all HTML elements, like this:
\begin{lstlisting}[style=htmlcssjs]
<section>
  <p>This is a paragraph.</p>
  <p>This is a paragraph.</p>
</section>
\end{lstlisting}



\item \bfb{Always Specify alt, width, and height for Images}: Always specify the \verb|alt| attribute for images. This attribute is important if the image for some reason cannot be displayed.

Also, always define the \verb|width| and \verb|height| of images. This reduces flickering, because the browser can reserve space for the image before loading.
\begin{lstlisting}[style=htmlcssjs]
<img src="html5.gif" alt="HTML5" style="width:128px;height:128px">
\end{lstlisting}

\item \bfb{Spaces and Equal Signs}: HTML allows spaces around equal signs. But space-less is easier to read and groups entities better together.

\begin{lstlisting}[style=htmlcssjs]
<link rel="stylesheet" href="styles.css">
\end{lstlisting}

\item \bfb{Place external style sheets within the} \verb|<head>| \bfb{tag}: Although external style sheets can be placed anywhere in the HTML document, it is best practice to place them within the <head> tag. This will allow your page to load faster.

\begin{lstlisting}[style=htmlcssjs]
<head>
    <link rel="stylesheet" href="style.css">
</head>
\end{lstlisting}


\end{itemize}


\subsection{CSS Best Practices}

\begin{itemize}
\item \bfb{Use proper spacing and indentation}: A CSS block should look like this:
\begin{lstlisting}[style=htmlcssjs]
selector {
    property: value;
}
\end{lstlisting}


\item \bfb{Use Hyphen Delimited Strings}: Use hyphens to separate two words: \verb|red-box| instead of \verb|redBox|. 

\item \bfb{Organize from general to specific}: Generally you should start with the most general selectors at the top and work your way down to the more specific selectors.  

\item \bfb{Avoid inline styles}: If at all possible write the CSS in the CSS-file. Try to keep HTML and CSS separate.
\end{itemize}

\subsection{JavaScript Best Practices}

As we learn JavaScript we will learn more about best practices and style guides. 

\begin{itemize}
\item \bfb{Use camelCase}: Multiple word variables should be written in camelCase.  This means that the first word is lower case and further words are capitalized: e.g. firstName, fullPrice, listOfValues

\item \bfb{Spaces Around Operators}: There should be a space around operators in JavaScript: e.g. i = 0.
\end{itemize}


\newpage

\section{HTML and CSS - A Short Repetition}

To get you back up to speed on HTML and CSS, please work through the following exercises.

Please take care to use proper syntax and formatting (indentation, spacing etc.)

\begin{ex}
\begin{enumerate}
\item Download and unzip the file \emph{personal-webpage.zip} (OneNote). 
\item Change the heading of the webpage and name to something personal.
\item Add a title to the webpage in the \verb|head|.
\item Change the picture of the website.
\item Give your picture an \verb|id| and use that id to style the border using CSS.
\item Add a link to your favorite website.
\item Add an image that is also a link to another website.
\item Make all \verb|h1| tags have text color red.
\item Change the background color to your favorite color. You can use \url{https://color.adobe.com/create/color-wheel} as a reference for creating a color scheme for your website.
\item Add a list of hobbies to the website.  Give the list the class \verb|hobbies|.
\item In the CSS file style the hobbies list to your liking.
\end{enumerate}
 
\end{ex}

\newpage

\section{Designing a Website -- from Layout to Final Product}

A key aspect of creating a website is planning the layout and design. In this chapter we look at how to select a color scheme for a website, design the large scale layout of the website and how to make nice backgrounds.

\subsection{Select a Color Scheme}
For finding colors that work well together we can use the Adobe color palette. -- or something similar (\#notanad): \url{https://color.adobe.com/create/color-wheel}.  In This example I selected a variety of complementary colors:
\centfig{0.5}{color-palette2}
We will note the corresponding RGB values to use later: 

\centfig{0.8}{rgb}


This does not mean that we need to use these colors exclusively in our website, but generallly these are colors that may work well together.

\subsection{Backgrounds}
There are two main ideas for backgrounds: We can use colors or an image. 

In CSS there are the following controls for the backgrounds: 
\begin{itemize}
\item \verb|background-color|
\item \verb|background-image|
\item \verb|background-repeat|
\item \verb|background-attachment|
\item \verb|background-position|
\end{itemize}
For in depth details you can see the reference guide at \url{https://www.w3schools.com/css/css_background.asp}. 

The \verb|background-color| property does pretty much what you would expect: 
\begin{lstlisting}[style=htmlcssjs]
body { 
	background-color: rgb(34,163,154);
}
\end{lstlisting}
Will create a solid background with the selcted color (one from our color wheel). In some instances this may produce a color that is too intense. We can 'dampen' the color by setting an alpha value: 
\begin{lstlisting}[style=htmlcssjs]
body {
    background-color: rgba(34, 163, 154,0.4);
}
\end{lstlisting}
You can also create gradients. Gradients are classified as images in CSS, so the correct way is as follows:
\begin{lstlisting}[style=htmlcssjs]
body {
    height: 100%;
    background-repeat: no-repeat;
    background-attachment: fixed;
    background-image: linear-gradient(#4AF0E4, #F0A883);
}
\end{lstlisting} 
For more details visit: \url{https://www.w3schools.com/css/css3_gradients.asp}

To insert an image, as we saw, we can use the \verb|background-image| property. This also often requires us to define whether the image should be repeated or where the image should be placed on the webpage.

For full details visit: \url{https://www.w3schools.com/css/css_background_image.asp}

cool backgrounds: \url{https://coolbackgrounds.io/}

\subsection{Layout}

There are different ways to achieve desired layouts. We will look at the one that I find most intuitive. Shown below are some examples of possible layouts:

%\begin{multicols}{3}
%\centfig{0.9}{page-example-1}
%
%\centfig{0.9}{page-example-2}
%
%\centfig{0.9}{page-example-3}
%
%\end{multicols}

All of them can be created with the \verb|grid| property in CSS.  We start by placing the individual sections in a "container" \verb|div|: 
\begin{lstlisting}[style=htmlcssjs]
<div class="container">
            <header>
                <h3>Header</h1>
            </header>
            <article>
                my Blog...
            </article>
            <nav>
                <ul>
                    <li>nav</li>
                    <li>nav</li>
                    <li>nav</li>
                    <li>nav</li>
                </ul>
            </nav>
            <footer>
                contact info etc.
            </footer>
        </div>
\end{lstlisting}

\verb|header|, \verb|article|, \verb|nav| and \verb|footer| are predefined in HTML. They are standard divisions of webpages and have the same functionality as \verb|divs|.

We start by naming the grid areas: 
\begin{lstlisting}
header {
    grid-area: header;
}

article {
    grid-area: article;
}

nav {
    grid-area: nav;
}

footer {
    grid-area: footer;
}
\end{lstlisting}
This just tells the grid what we want to call each element. Here the names of the grid areas are the same as the \verb|div| names, but they could be anything as long as you're consistent.

We then define the grid and layout of the container: 
\begin{lstlisting}
.container {
    display: grid;
    grid-template-columns: 1fr 5fr;
    grid-template-rows: 1fr 8fr 1fr;
    grid-template-areas: 
    'header header'
    'nav article'
    'nav footer';
    gap: 1em;
    min-height: 100vh;
}
\end{lstlisting}
With the following functionalities: 
\begin{center}
\begin{tabular}{p{0.5\textwidth}|p{0.5\textwidth}}
\verb|display: grid;| & creates a grid layout \\
\hline
\verb|grid-template-columns: 1fr 5fr;| & splits the columns in a $1:5$ ratio \\
\hline
\verb|grid-template-rows: 1fr 8fr 1fr;| & splits the rows in a $1:8:1$ ratio \\
\hline
\begin{verbatim}
grid-template-areas: 
    'header header'
    'nav article'
    'nav footer';
\end{verbatim} & header spans both columns,  
nav spans the two lower rows 
and article and footer both span one row and one column.  \\
\hline
\verb|gap: 1em;| & places a gap of 1em (space for a letter m) around each grid element\\
\hline
\verb|min-height: 100vh;| & makes the page fill the full view height.
\end{tabular}
\end{center}

\subsection{Exercises}
\begin{ex}
Copy the code to your personal workspace. 
\end{ex}
\begin{ex}
Use the color wheel to select an assortment of colors for your website. Cope the RGB values so you can use them.
\end{ex}
\begin{ex}
Make the background of your webpage a single color, a gradient and an image.
\end{ex}
\begin{ex}
Change the layout options in the \verb|container| selector to create the other two example layouts. You will have to figure out how to add the 'logo'.
\end{ex}

\begin{ex}
Use all of these ideas to create your dream website\ldots
\end{ex}


\newpage
\section{SVG -- Scalable Vector Graphics}

Most of the images  and graphics you use and find online are based on pixels. The downside to these graphics are that they do not scale nicely -- i.e. if you zoom in enough, you will always see the individual pixels eventually. Vector graphics are based on an entirely different idea: instead of telling the computer which pixels to turn on and off, we tell the computer what shape we want and allow it to fill in the pixels as necessary. This means that no matter how zoomed in or out we are, the computer always knows what the image should look like and can draw it with the appropriate resolution.

\subsection{SVG in HTML}
Scalable Vector Graphics (SVG) can be thought of as a series of commands that tell your computer what do draw on your screen. They are easily embedded into HTML code, but can also be used as separate image files (.svg).



We start with a very simple example: 
\begin{lstlisting}[language=html]
<svg height="100" width="100">
    <circle cx="50" cy="50" r="50" fill="red" />
</svg>
\end{lstlisting}
As you can see the code looks a lot like HTML: We  have a \verb|svg| tag together with an assortment of attributes. We will now look at some basic shapes that we can draw using svg. We can write the \verb|svg| tag directly inside an html file. The width and height attribute of the svg tag denote how much space on the webpage should be reserved for that image.

{\bf Remark.} Note that the tags for the basic shapes are unpaired, so we finish them with \verb|/>|.


\newpage

\subsubsection{Circles}
The \verb|<circle>| element draws a circle on the screen. It takes 3 basic parameters to determine the shape and size of the element:
\begin{center}
\begin{tabular}{l|p{15cm}}
\verb|r| & The radius of the circle. \\
\verb|cx| & The x position of the center of the circle. \\
\verb|cy| & The y position of the center of the circle. 
\end{tabular}
\end{center}
Apart from that there are also a number of other attributes that can be set. These attributes can be set for any svg shape.
\begin{center}
\begin{tabular}{l|p{15cm}}
\verb|fill| & The fill color. \\
\verb|stroke| & The stroke color. \\
\verb|stroke-width| & The width of the stroke. \\
\verb|opacity| & The opacity of the shape. 
\end{tabular}
\end{center}

So the code on the left will produce the image on the right:
\begin{multicols}{2}
\begin{lstlisting}[language=html]
<circle
        cx="50"
        cy="50"
        r="40"
        stroke="green"
        stroke-width="4"
        fill="yellow"
        opacity="0.5"
        />
\end{lstlisting}
\columnbreak

\centfig{0.7}{circle1.png}
\end{multicols}

\begin{ex}
Draw three circles in a row, that touch each other and get smaller from left to right. The circles should all have different fill colors.
\end{ex}

\subsubsection{Rectangles} 
 
 The \verb|<rect>| element draws a rectangle on the screen. There are 6 basic attributes that control the position and shape of the rectangles on screen. 
 \begin{center}
 \begin{tabular}{l|p{15cm}}
\verb|x| & The x-position of the top left corner. \\
\verb|y| & The y-position of the top left corner. \\
\verb|width| & The width of the rectangle. \\
\verb|height| & The height of the rectangle.  \\
\verb|rx| & The x-radius for rounded corners (if this is not set it defaults to 0). \\
\verb|ry| & The y-radius for rounded corners (if this is not set it defaults to 0).
\end{tabular}
\end{center}

{\bf An example:}
\begin{multicols}{2}
\begin{lstlisting}
      <rect
        x="30"
        y="80"
        width="200"
        height="50"
        fill="blue"
        stroke="red"
        opacity="0.5"
        stroke-width="4px"
      />
\end{lstlisting}
\columnbreak

\centfig{0.8}{rect1.png}
\end{multicols}


\newpage
\subsubsection{Ellipses}

An \verb|<ellipse>| is a more general form of the <circle> element, where you can scale the x and y radius (commonly referred to as the semimajor and semiminor axes in maths) of the circle separately.
\begin{center}
\begin{tabular}{l|p{15cm}}
\verb|rx| & The x-radius of the circle. \\
\verb|ry| & The y-radius of the circle. \\
\verb|cx| & The x position of the center of the circle. \\
\verb|cy| & The y position of the center of the circle. 
\end{tabular}
\end{center}

{\bf An example:}
\begin{multicols}{2}
\begin{lstlisting}
<ellipse 
      cx="30"
      cy="60"
      rx="20"
      ry="50"
      fill="purple"
      />
\end{lstlisting}
\columnbreak

\centfig{0.3}{ellipse1.png}
\end{multicols}


\newpage
\subsubsection{Lines}
The \verb|<line>| element takes the positions of two points as parameters and draws a straight line between them.
\begin{center}
\begin{tabular}{l|p{15cm}}
\verb|x1| & The x position of  point 1. \\
\verb|y1| & The y position of  point 1. \\
\verb|x2| & The x position of  point 2. \\
\verb|x2| & The y position of  point 2. 
\end{tabular}
\end{center}

{\bf An example:}
\begin{multicols}{2}
\begin{lstlisting}
<line 
      x1="10" 
      x2="50" 
      y1="110" 
      y2="150" 
      stroke="black" 
      stroke-width="5"
      />
\end{lstlisting}
\columnbreak

\centfig{0.3}{line1.png}
\end{multicols}

\newpage
\subsubsection{Polylines}
A \verb|<polyline>| is a group of connected straight lines. Since the list of points can get quite long, all the points are included in one attribute.
\begin{center}
\begin{tabular}{l|p{15cm}}
\verb|points| & List of points
\end{tabular}
\end{center}

Each number must be separated by a space, comma, EOL ("end of line"), or a line feed character. Each point must contain two numbers: an x coordinate and a y coordinate. So, the list \verb|(0,0)|, \verb|(1,1)|, and \verb|(2,2)| could be written as \verb|0, 0 1, 1 2, 2|. You can also write each point on a new line for clarity:

{\bf An example:}
\begin{multicols}{2}
\begin{lstlisting}
<polyline 
      points="
      60, 110 
      65, 120 
      70, 115 
      75, 130 
      80, 125 
      85, 140 
      90, 135 
      "
      fill="none"
      stroke="red"
      stroke-width="2"
      />
\end{lstlisting}
\columnbreak

\centfig{0.8}{polyline1.png}
\end{multicols}

Note that by deafult the line has a fill color, which may not be desired. \verb|fill="none"| eliminates this. Also take note that the points are enclosed by quotation marks.

\newpage
\subsubsection{Polygons}
A \verb|<polygon>| is similar to a \verb|<polyline>|, in that it is composed of straight line segments connecting a list of points. For polygons though, the path automatically connects the last point with the first, creating a closed shape.

{\bf An example:}
\begin{multicols}{2}
\begin{lstlisting}
<polygon 
      points="
      200,10 
      250,190 
      160,210
      " 
      fill="lime"
      stroke="purple"
      stroke-width="1" 
      />
\end{lstlisting}
\columnbreak

\centfig{0.5}{polygon1.png}
\end{multicols}

\begin{ex}
Draw a five-sided star.
\end{ex}

\begin{ex}
Draw a regular octagon.
\end{ex}

\newpage



\subsubsection{Paths}
A \verb|<path>| is the most general shape that can be used in SVG. Using a path element, you can draw rectangles (with or without rounded corners), circles, ellipses, polylines, and polygons. Basically any of the other types of shapes, bezier curves, quadratic curves, and many more.

To full understand paths requires a lot of additional knowledge. If you want to find out more about paths, you can follow the tutorial at

\url{https://developer.mozilla.org/en-US/docs/Web/SVG/Tutorial/Paths}

For now just know that a path is defined by a single attribute:
\begin{center}
\begin{tabular}{l|p{15cm}}
\verb|d| & A list of points and other information about how to draw the path.
\end{tabular}
\end{center}

{\bf An example:}
\begin{multicols}{2}
\begin{lstlisting}
<path 
      d="
      M20,230 
      Q40,205 
      50,230 
      T90,230
      " 
      fill="none" 
      stroke="blue" 
      stroke-width="5"
      />
\end{lstlisting}
\columnbreak

\centfig{0.5}{path1.png}
\end{multicols}

\newpage
\subsection{Animations}
The modern internet loves animation. With SVG these can be created relatively easily. We will not look at the deep theory of animation, however that wont stop us from playing around with it a little:

\begin{lstlisting}
<svg height="100" width="100">
    <ellipse cx="50" cy="50" rx="50" ry="50" fill="red">
        <animate attributeName="ry"
            values="50;25;50"
            dur="2s"
            repeatCount="indefinite" />
    </ellipse>
</svg>
\end{lstlisting}

{\bf Remark.} Note that \verb|<animate>| is one of the few tags that can exist inside another tag.

\begin{ex}
Create a smiley face "emoji", that winks one of its eyes.

Hint: A curved path for the smile could have the following code:
\begin{lstlisting}
    <path 
    d="M 0 100 Q 50 150 100 100" 
    stroke="black" 
    stroke-width="2" 
    fill="none"
    />
\end{lstlisting}
\verb|(0,100)| ist the left corner, \verb|(100,100)| is the right corner and \verb|Q 50 150| determines how big (how far down) the smile is.
\end{ex}

\begin{ex}
In groups of two, draw your own animation. We will show them at the end of the lesson.
\end{ex}

\newpage
\subsection{Bonus: Animate Motion}

The SVG \verb|<animateMotion>| element provides a way to define how an element moves along a motion path.

\begin{lstlisting}
<svg viewBox="0 0 200 100" xmlns="http://www.w3.org/2000/svg">
  <path
    fill="none"
    stroke="lightgrey"
    d="M20,50 C20,-50 180,150 180,50 C180-50 20,150 20,50 z" />

  <circle r="5" fill="red">
    <animateMotion
      dur="10s"
      repeatCount="indefinite"
      path="M20,50 C20,-50 180,150 180,50 C180-50 20,150 20,50 z" />
  </circle>
</svg>
\end{lstlisting}

\newpage
\section{HTML Inputs}

In this section we will be looking at how we can read and process user inputs from a website. I'm pretty sure you have all had to fill in  a webform at some point. We will be looking at the building blocks we need to interact with the user of a website.

\subsection{A First Example}
\begin{ex}
Navigate to the page \url{https://olidec.github.io/html-input/} and select \verb|A Short Game|. Together with a partner fill in the requested words in the fields (One person asks and types the replies) then click \verb|Submit|. 
\end{ex}


\subsection{Getting Started}

\renewcommand{\labelenumi}{\arabic{enumi}.}
\begin{enumerate}
\item Downoad the code of the website.
\begin{itemize}
\item Option 1: Download the zip file from OneNote (HTML Inputs -> Zip File....)
\item Option 2: Navigate to \url{https://github.com/olidec/html-input} and click \verb|Code -> Download ZIP|.
\end{itemize}
\item Move the zip file to your Computer Science folder and unzip (\verb|right-click -> extract|. You should now have a new subfolder in your CS folder called \verb|html-input-master|.
\item Open Visual Studio Code select \verb|file -> Open Folder...| and navigate to the previously created folder and select. You should now have a workspace that looks roughly like this:
\centfig{0.4}{workspace1.png} 
\item If you have not yet done so, select the extensions panel on the left hand side and search for 'live server'. Install the version by \emph{Ritwick Dey}. This will allow us to immediately see the changes we make in our html code.
\end{enumerate}

\begin{ex}
Open the file \verb|madlib1.html| and together with a partner try to replace the text with a different one (you can find some examples at \url{https://www.madtakes.com/index.php} or by searching for 'madlibs'.  Make sure to adapt the 'names' of the input fields to what you are looking for. Try to add additional input fields and access them via the script.

Find someone who does not know your text and play through the game with them. Have fun.
\end{ex}


\subsection{The Basics}

In HTML an input field is defined by the \verb|<input>| tag. There are a whole range of different types of inputs. The entire list of options together with some examples can be found at
\begin{center}
\url{https://www.w3schools.com/html/html_form_input_types.asp}
\end{center}

Click on the small \verb|file +| button in the explorer pane and create a new file called \verb|input-test.html|. In the new file write \verb|html:5| and push the enter key. This will give you a basic html5 template.

We will start with a text input field: 
\begin{verbatim}
<input type="text" id="my-text-input">
\end{verbatim}
The \emph{type} tells the website what type of input to expect and the \emph{id} will allow us to access that input later on. This already creates an input field on our website. If we want some instructions next to or below the input field we can just add the text after the \verb|<input>| tag:
\begin{verbatim}
<input type="text" id="my-text-input"> Type your input here.
\end{verbatim}

The input field will not be able to do anything unless we have some sort of submit-button. This is added via the \verb|<button>| tag (We add the \verb|<br>| to put the button under the input field): 
\begin{verbatim}
<input type="text" id="my-text-input"> Type your input here. <br>
<button type="button">Submit</button>
\end{verbatim}

An submit button is pretty useless, if it doesn't do anything so in the next section we will look at how to make the button actually do something.

\subsection{A First Interaction}

Interactions on websites are done via so-called \emph{scripts}. There are different languages in which these can be written; ours will be written in \emph{JavaScript} or \emph{JS} for short.  We will be doing a full introduction to JavaScript as well but for now we will only look at the code we actually need to create some interactivity on our website.

The main idea is the following: When we click on a button, the webpage will access a \emph{function} that is written in the script.  We do this by adding an \verb|onclick| attribute to the button: 
\begin{verbatim}
<input type="text" id="my-text-input"> Type your input here. <br>
<button type="button" onclick="myFunction()">Submit</button>
\end{verbatim}
Note that functions need input brackets \verb|()| even if there are no arguments and that we write the names in camelCase (JS convention).

We now have to write a function that does \emph{something} (i.e. whatever we tell it to do). In order to do this we will add a new file called \verb|my-script.js|.  In order for the website to know where to look for the script we have to add the following line to the \verb|<head>| of our document:
\begin{verbatim}
<script src="my-script.js"></script>
\end{verbatim}
This tells the website to look for a file called \verb|my-script.js| and access any code that is in there.

\subsection{Accessing the Input}

To access the input we need what are known as \emph{selectors}. They all have a similar structure and we will not go into too much detail on how they work. We just need to be able to use them in our application.

A commonly used selector selects an HTML element by its \verb|id|: 
\begin{verbatim}
document.getElementById("my-text-input").value;
\end{verbatim}
Let's break this down: This selector goes through the (HTML-) document and looks for an element with \verb|id="my-text-input"| it then selects the \emph{value} of that element. In this case the value is whatever we have typed in that input field.

We now have to do something with that input value. We can have some text below the input field and  add in the input value. To do this we will write:
\begin{verbatim}
<h2 id="my-input-value">Your Text will go Here</h2>
\end{verbatim}
We can select this element as follows:
\begin{verbatim}
document.getElementById("my-input-value").innerHTML;
\end{verbatim}
This accesses the \emph{inner HTML} part i.e. the part that is between the opening and closing tags.

Using all of this we can now write our function: In the my-script.js file write the following code:
\begin{verbatim}
function myFunction() {
    const myinput = document.getElementById("my-text-input").value;
    document.getElementById("my-input-value").innerHTML = myinput;
}
\end{verbatim}

This assigns the input value to the variable called \verb|myinput| and the replaces the inner HTML of the target (a \verb|<h2>| header) with that input.

You have now created your first interactive website.

\begin{ex}
Open the site \url{https://olidec.github.io/html-input/} and click on \verb|HTML Inputs|. Play around with the different input types and look at  and try to understand the code. 
\end{ex}

\begin{ex}
Have a look at the website \url{https://www.w3schools.com/html/html_form_input_types.asp}

In your \verb|input-test.html| file try to add two new input fields with types that we have not yet used. Try to adapt the script so that it can work with these new inputs.
\end{ex}

\begin{ex}
Use CSS to style your \verb|input-test.html| page. Remember to add a link to your external stylesheet in the \verb|<head>| of your document.
\end{ex}


\newpage

\section{Introduction to JavaScript}

JavaScript (JS) is the world's most popular programming language. It is one of the three main components of modern web design:
\begin{itemize}
\item HTML for the \emph{content} of a website,
\item CSS for the \emph{styling} and \emph{layout},
\item and JavaScript for \emph{interactivity} and \emph{responsiveness}.
\end{itemize}

We will be using a graphical "turtle" environment to get started. Later on we will create responsive websites and even games.

\subsection{Getting Started}

\begin{enumerate}
\item Download the necessary files to your personal CS folder. Either via OneNote or \url{https://github.com/olidec/turtles-svg-js}.
\item Create a new file called \verb|moves.js|. We will write all of our code in here to begin with.
\item For your first piece of code, write the following in to \verb|moves.js|:
\begin{verbatim}
moveForward(200);
turn(90);
moveForward(200);
turn(90);
moveForward(200);
turn(90);
moveForward(200);
\end{verbatim}
and save.
\item Open \verb|index.html| with Live Server -- either by clicking on the bottom menu in VS Code or by right-clicking on the file and selecting "Open with Live Server".
\item Click the button "Draw Your Moves".
\end{enumerate}

Congratulations you have written your first program. A program is a set of instructions that get executed by a computer or processor. In this case we are writing in JavaScript (although the instructions are not standard) and the code gets executed by our webbrowser.

\subsection{Turtle Basics}

The turtle graphics we use have a set of instructions that are built in. These can also be combined with regular JavaScript. Note that all of these instructions are functions -- so they are always written with brackets even if there is no input.
\renewcommand{\arraystretch}{1.5}
\begin{center}
\begin{tabular}{p{3cm}|p{8cm}}
\verb|penDown()| & Places the "pen" on the drawing ares. Any movements made from now on will be connected by a line. \\
\hline
\verb|penUp()| & The "pen" is lifted from the drawing area. Movements will not be connected by a line.  \\
\hline
\verb|moveForward(d)| & Move the "turtle" forward by \verb|d| units (think \emph{distance}).  \\
\hline
\verb|moveTo(x,y)| & Moves the "turtle" to an exact set of coordinates \verb|(x,y)|. \\
\hline
\verb|turn(a)| & Rotates the "turtle" by \verb|a| degrees (think \emph{angle}). Note that angles are measured counter-clockwise from the direction the turtle is facing. 
\end{tabular}
\end{center}
The graphing window has a width and height of \verb|1000| units and the coordinates go from \verb|-500| to \verb|500| in both  the \verb|x|- and \verb|y|-direction. The turtle starts at \verb|(0,0)| and faces to the right.

{\bf Remark.} Note that the turning angles are defined couter-clockwise (to the left). If we want our turtle to turn to the right, we need to set a negative angle.

\begin{ex}\label{ex17}
\begin{multicols}{2}
Write code that draws a regular hexagon.

\centfig{0.4}{hexagon.png}
\end{multicols}
\end{ex}
\sol{

moveForward(100)\\
turn(60)\\
moveForward(100) \\
turn(60)\\
moveForward(100)\\
turn(60)\\
moveForward(100)\\
turn(60)\\
moveForward(100)\\
turn(60)\\
moveForward(100)
}

\begin{ex}\label{ex18}
\begin{multicols}{2}
Write code that draws a five-sided star. 

\centfig{0.4}{5-star.png}
\end{multicols}
\end{ex}
\sol{

moveForward(100) \\
turn(144) \\
moveForward(100)\\
turn(144)\\
moveForward(100)\\
turn(144)\\
moveForward(100)\\
turn(144)\\
moveForward(100)
}

\begin{ex}\label{ex19}
What image does the following code produce. Make a guess before copying the code to your \verb|moves.js| file.
\begin{verbatim}
moveForward(100);
turn(90);
moveForward(100);
turn(-90);
moveForward(100);
turn(90);
moveForward(100);
turn(-90);
moveForward(100);
\end{verbatim}
\end{ex}
\sol{
A 'staircase' with three 'steps'.
}




\subsection{Repetition (Loops)}

As we saw in the previous examples, we often have to reuse the same code or multiple steps again and again.  Because coders are lazy and do not want to write more than absolute necessary, they invented a cool thing called a \emph{loop}: We give a set of instructions and tell the computer how often to repeat them. 

The syntax for a \emph{for loop} in JavaScript is as follows:
\begin{verbatim}
for (initial-expression; condition; second-expression) {
	a set of instructions
}
\end{verbatim}
The \verb|initial-expression| tells the computer where to start, the \verb|second-expression| instructs the computer what to do after it has completed one execution of the instructions and the \verb|condition| tells the computer when to stop. If only Mickey had set a proper exit condition, he would not have had so much trouble:
\centfig{0.6}{mickey}

The code for the square from our first example can now be written as a loop:
\begin{verbatim}
for (let i = 0; i <= 3; i++) {
    moveForward(200);
    turn(90);
}
\end{verbatim}
Copy this code to your moves and verify that it draws the square properly.

Breakdown:
\begin{center}
\begin{tabular}{p{3cm}|p{8cm}}
\verb|for| & Initialises a \emph{for loop}. The computer will be expecting certain kinds of instructions.\\
\hline
\verb|let i = 1| & We set a \emph{variable} to have the value 0. Computer scientists like to start counting at 0.  This variable is often called the \emph{counter}.\\
\hline
\verb|i < 4| & As long as \verb|i| is less than 4 the instructions will be executed. \\
\hline
\verb|i++| & After one set of instructions is executed, the counter is increased by 1. 
\end{tabular}
\end{center}

\coloredbox{
{\bf Best Practice:} It is not necessary to indent the code chunk inside of a for loop, but it makes the code easier to read.
}

\begin{ex}
Rewrite exercises \ref{ex17} and \ref{ex18} using the for loop as shown above.
\end{ex}
\sol{

for (let i = 0; i < 6; i++) \{ \\
\quad moveForward(100) \\
\quad turn(60) \\
\}

and 

for (let i = 0; i < 5; i++) \{ \\
\quad moveForward(100) \\
\quad turn(144) \\
\}
}


\begin{ex}
(*optional challenge) Rewrite exercise \ref{ex19} using a for loop as shown above. 
\end{ex}
\sol{

for (let i = 0; i < 5; i++) \{ \\
 moveForward(100) \\
 turn(90*(-1)**i) \\
\}
}

\subsection{Variables}
A variable in computer science is a placeholder for a specific value. If we want a variable to changed during the running of our code, we use \verb|let| or \verb|var| to give it a name. If the value should always be the same we should use \verb|const|.  

\coloredbox{
{\bf Best Practice:} If possible use \texttt{let} to define your variables, if they can change. Use \texttt{const} if they stay the same.
}

\newpage
Let's look at the following code chunk:
\begin{verbatim}
let d = 100;

for (let i = 0; i < 5; i++) {
	moveForward(d);
	turn(72);
}
\end{verbatim}

Here we have set a variable \verb|d| to the value 100. If we now want to change the length of the sides, we can just change it at the beginning of the document. This is especially useful if the same value is used in multiple places -- we can just change it once and all the instances will be changed.

\coloredbox{
{\bf Best Practice:} Use camelCase to name your variables and use variable names that describe what they represent.
}

So a "better" version of the code above might be:
\begin{verbatim}
let sideLength = 100;
let angle = 72;

for (let i = 0; i < 5; i++) {
	moveForward(sideLength);
	turn(angle);
}
\end{verbatim}

\newpage
Using loops and variables we can now write code that produces a "random walk": 
\begin{verbatim}
let d = 10;
for (let i = 0; i < 1000; i++) {
    moveForward(d);
    let a = Math.floor(Math.random()*3)*90-90;
    turn(a);
}
\end{verbatim}
The command \verb|Math.random()| creates a random number between 0 and 1. The other bits are some math to give us an angle of either -90$\dg$, 0$\dg$ or 90$\dg$.  In each step the turtle turns to a random one of these values and then moves forward by 10 units. 

\subsubsection{The Math Bits}
Random numbers from the \verb|Math.random()| function give an output between 0 and 1. So for example if we want a value between 0 and 10 , we can just multiply that output by 10: \verb|Math.random()*10|. Now if we want values between 10 and 20, we can just add 10 to every output: \verb|Math.random()*10+10|. In general, if we want values between $a$ and $b$, then we can use \verb|Math.random()*(b-a)+a|.

If we only want integer values, then we use the \verb|Math.floor()| function. This always rounds down to the next lower integer. So to get whole numbers between 0 and 2, we multiply the random output by 3 and then round down: \verb|Math.floor(Math.random()*3)|. If we then multiply this result by 90 and subtract 90, will get a random value of -90$\dg$, 0$\dg$ or 90$\dg$.

\subsection{Functions}

Continuing with the idea that programmers are lazy, we also need the concept of a \emph{function}. A function is a specific set of instructions that we may want to execeute multiple times in our program. It is also possible to pass different parameters or arguments to the function.

Let's say we want to draw multiple squares, but each one slightly larger than the previous one. We can do this by first defining a function that draws a square with the length of the side as a parameter: 

\begin{verbatim}
function drawSquare(s) {
    for (let i = 0; i < 4; i++) {
        moveForward(s);
       turn(90);
    }
}
\end{verbatim}
We can see the parameter \verb|s| gets plugged in to the \verb|moveForward()| function. We can now call this function mulitple times: 
\begin{verbatim}
drawSquare(50);
drawSquare(100);
drawSquare(150);
drawSquare(200);
\end{verbatim}

{\bf Remark. } We can even simplify those lines above with a \verb|for|-loop:
\begin{verbatim}
for (let i = 0; i < 4; i++) {
    drawSquare(50*(i+1));
}
\end{verbatim}

As we can see, the counter \verb|i| can also be used within the loop as part of a formula.
\begin{ex}
Copy the code above and regard the output.
\end{ex}

\begin{ex}
\begin{multicols}{2}
Write code that produces the image to the right.

\centfig{0.4}{hexagons}
\end{multicols}
\end{ex}
\sol{

function drawHexagon(s) \{

    for (let i = 0; i < 6; i++) \{
    
        moveForward(s)
        
        turn(60)
        
    \}
    
\}

drawHexagon(50)

drawHexagon(100)

drawHexagon(150)

drawHexagon(200)
}





\subsection{Decisions (Conditions)}

Often times in programming we need to make a decision and proceed depending on the previous result.  The standard way to do this is via an \verb|if-else| statement.  The syntax for a conditional statement is as follows:
\begin{verbatim}
if (condition) {
    do this
}
\end{verbatim}
Sometimes we also want to tell the computer what to do if the condition is not met:
\begin{verbatim}
if (condition) {
    do this
}
else {
    do that
}
\end{verbatim}
For example, the following code draws a random walk and decides, whether the path will leave a circle of radius 200. If if does it stops with the command \verb|break| otherwise it continues to draw the path:
\begin{verbatim}
let d = 10;
let n = 500;
for (let i = 0; i < n; i++) {
    moveForward(d);
    if (position.x**2 + position.y**2 > 200**2) {
        document.getElementById("turtle").setAttribute("fill","red");
        break
    }
        else {
            let a = Math.floor(Math.random()*3)*90-90;
            turn(a);
        }
}
\end{verbatim}

\begin{ex}
Read and understand the code above. 
\begin{enumerate}
\item What is \verb|d|?
\item What is \verb|n|?
\item What are \verb|position.x| and \verb|position.y|?
\item Copy it to your moves file and run it. Does your turtle turn red?
\end{enumerate}
\end{ex}
\sol{
\begin{enumerate}
\item The distance travelled by the turtle in each step.
\item The number of steps taken by the turtle.
\item The $x$- and $y$-coordinates of the turtle.
\end{enumerate}
}

\newpage

\subsection*{Solutions}
\printcursols



\newpage
\section{A Small Quiz}
In this section we look at a small quiz that we can write using HTML inputs and JavaScript. We first need some new bits and pieces:
\subsection{JSON}
JSON Stands for "JavaScript Object Notation". An Object in Computer Science is a very important concept, but we will delve into that later. The important thing to know is that a \emph{JSON Object} consists of "key-value" pairs:
\begin{verbatim}
const person = '{"name":"Oliver De Capitani", "age":41, "job":"teacher"}';
\end{verbatim}
In order for JavaScript to be able to read a JSON object, we need to parse it (\emph{parsing} is generally what we call translating something for someone else -- in this case for JavaScript).
\begin{verbatim}
const parsedPerson = JSON.parse(person);
\end{verbatim}
We can now access the values by using the keys: \verb|parsedPerson.age| returns 41 etc.

\subsection{Template Literals}
Template literals or template strings are literals delimited with backtick (`) characters. We can use these to pass variables into text (and many other things). To do this we enclose our variable in curly brackets preceded by a dollar sign: \verb|${...}|.

{\bf Example.} Add a \verb|div| with the id \verb|input| into our html document:
\begin{verbatim}
<div id="input"></div>
\end{verbatim}
\newpage
We can now write our JSON input directly to this \verb|div|:
\begin{verbatim}
document.getElementById("input").innerHTML = 
`
${parsedPerson.name} is ${parsedPerson.age} years old 
and is a ${parsedPerson.job}
`;
\end{verbatim}

\subsection{Arrays}
Arrays are collections of objects or values. In JavaScript arrays are delimited by square brackets \verb|[...]|. Array values can be accessed via their \emph{index} that starts at zero:
\begin{verbatim}
data = [a,b,c,d]
\end{verbatim}
To get the first entry we use \verb|data[0]|, to get the second one \verb|data[1]| etc. If we go too far, we will get an error message: e.g. \verb|data[4]| will not be able to return a value.

We can also find the length of an array with \verb|data.length|, this is a built in method and, in this case, will return 4.

\subsection{The HTML}
The HTML sets up the basics blocks needed to show the quiz. The individual elements are distinguished via ids. The code is available on OneNote.

\subsection{The JavaScript}
The JavaScript contains the questions and the means for displaying them and also for checking the answers. The code is available on OneNote.


\begin{ex}
Copy the HTML and JS files to your folder and play through the quiz.
\end{ex}

\begin{ex}
Add at least three questions of your own.
\end{ex}

\begin{ex}
Use CSS to style your quiz.
\end{ex}

\newpage

\section{A First Game}

\begin{ex}
\begin{enumerate}
\item Navigate to your CS folder and create a new subfolder called 'games'.
\item Within the 'games' folder create a new subfolder called 'res'.
\item Download and unzip the Resources from OneNote.
\item Move the unzipped images (only the resources) to your newly created folder 'img'.
\item Open Visual Studio Code and open the 'games' folder (navigate inside the 'games' folder so you see the 'res' folder.
\item Create files called \verb|pop.html|, \verb|pop.js| and \verb|pop.css|.
\item Copy the code from OneNote into the appropriate files. 
\item Open \verb|pop.html| with LiveServer and play the game.
\end{enumerate}
\end{ex}

\begin{ex} Look through the three files, answer the following questions and try to solve the extra exercises.
\begin{enumerate}
\item In \verb|pop.html| on line 12: Why is the \verb|<h1>| tag empty? Where is its content generated?
\item Which lines of code do you have to change to make the balloon larger or smaller? Try it and see if everything works.
\item What does the function \verb|clickedIt| do?
\item Change the message at the end of the game.
\item What happens if you change the value 1000 to 2000 in \verb|pop.js|?
\item What happens if you change the 0.2s to 6s in \verb|pop.css|? Does the game get easier or harder?
\item How long does the game last? Can you make the game go longer?
\item How does the balloon move around the webpage? Can you make the balloon move over more of the area? Can you restrict the balloon to the bottom right corner of the webpage?
\end{enumerate}
\end{ex}
\sol{
\begin{enumerate}
\item The contents are generated in the pop.js file on line 7.
\item Line 5 of pop.css and line 12 of pop.js.
\item Increases the score by one, shows the updated score and sets the balloon size to 1px (very small).
\item Change line 17.
\item The balloons change position every 2s instead of 1s.
\item The animation takes 6s -- much easier.
\item The game stops when the balloon has changed position 20 times. We can adjust by changing line 16 in pop.js
\item Restricted to bottom right corner: \\
    balloon.style.left = (Math.random()*400 + 400) + 'px'; \\
    balloon.style.top = (Math.random()*300 + 300) + 'px';
\end{enumerate}
}

\begin{ex}
\begin{enumerate}
\item Look at the code in the file \verb|tennis.pdf| on OneNote (or printout) and try to understand what it does.
\item In your 'games' folder create the files \verb|tennis.html|, \verb|tennis.js| and \verb|tennis.css|.
\item Copy the code from \verb|tennis.pdf| into the appropriate files. Make sure to use  the proper styling for your code (best practices). 
\item When you are finished, play the game (open \verb|tennis.html| with LiveServer).
\item Change the colors of the board, the ball and the bat.
\item What is the initial position of the ball?
\item Where does the bat start?
\item Change the \verb|border-radius| value to a smaller value. What happens?
\item Can you make the ball move more quickly?
\item Can you make the game board larger? (Remember to change when the ball bounces.)
\item Add a score variable to the game. 
\end{enumerate}
\end{ex}
\sol{
\begin{enumerate}
\item ok
\item ok
\item Make sure you include the files in the header with link:css and script:src. write 'defer' after  src="..." but before >.
\item ok
\item Change the css.
\item At (100,100)
\item At (0,420)
\item It no longer looks like a ball.
\item Change speedY (dropping speed). speedX changes the speed in the horizontal direction.
\item In the css. Remember to also change the lines with all the 'if' statements.
\item You can pretty much copy this from the previous example. You just have to add a <h1> tag with id="scoreBox" to the html-file.
\end{enumerate}
}


\begin{ex}
\begin{enumerate}
\item Look at the code in the file \verb|catch.pdf| on OneNote (or printout) and try to understand what it does.
\item Read the documentation of the \verb|canvas| at \url{https://www.w3schools.com/html/html5_canvas.asp}.
\item What does the function \verb|keyPressed| do? Which keys need to be pressed (Use the image on your OneNote)?.
\item In your 'games' folder create the files \verb|catch.html|, \verb|catch.js| and \verb|catch.css|.
\item Copy the code from \verb|catch.pdf| into the appropriate files. Make sure to use  the proper styling for your code (best practices). 
\item When you are finished, play the game (open \verb|catch.html| with LiveServer).
\item Change \verb|dodgerblue| to another color value.
\item Make the apple move faster.
\item Can you make the monster move more quickly?
\item Change the code so that each apple is worth 5 points.
\item Make the game work with different keys on the keyboard.
\item Make the game run for 1 minute instead of 30 seconds.
\end{enumerate}
\end{ex}


\newpage

\subsection*{Solutions}
\printcursols
\newpage
\section{Some More Turtle Exercises}

\begin{ex}
\begin{multicols}{2}
Can you write code that draws this shape in one single path? Hint: the square root of a distance \verb|d| can be calculated with \verb|Math.sqrt(d)|.

\centfig{0.5}{house}
\end{multicols}
\end{ex}
\sol{

moveForward(100)

turn(90)

moveForward(100)

turn(45)

moveForward(Math.sqrt(2)*50)

turn(90)

moveForward(Math.sqrt(2)*50)

turn(45)

moveForward(100)

turn(135)

moveForward(Math.sqrt(2)*100)

turn(135)

moveForward(100)

turn(135)

moveForward(Math.sqrt(2)*100)
}

\begin{ex}
Write code in the \verb|moves.js| file that changes the stroke color. Hint: \verb|document.getElementById(...)|.
\end{ex}
\sol{
document.getElementById("turtlepath").setAttribute("stroke","blue")
}

\begin{ex}
Use a loop to draw an equilateral triangle with sides of length 200.
\end{ex}
\sol{

for (let i = 0; i < 3; i++) \{

    moveForward(200)
    
    turn(120)
    
\}
}

\begin{ex}
Use a loop to draw a square with sides of length 300.
\end{ex}
\sol{

for (let i = 0; i < 4; i++) \{

    moveForward(300)
    
    turn(90)
    
\}
}

\begin{ex}
Use a loop to draw a regular pentagon with sides of length 300.
\end{ex}
\sol{

for (let i = 0; i < 5; i++) \{

    moveForward(300)

    turn(360/5)

\}
}

\begin{ex}
Use a loop to draw a regular \emph{dodecagon} (12-sided polygon) with sides of length 100.
\end{ex}
\sol{

for (let i = 0; i < 12; i++) \{

    moveForward(100)

    turn(360/12)

\}
}

\begin{ex}
Use a loop to draw a regular  17-gon with sides of length 50.
\end{ex}
\sol{

for (let i = 0; i < 17; i++) \{

    moveForward(50)

    turn(360/17)

\}
}

\begin{ex}
Write a function \verb|drawPolygon(n,d)| that draws a polygon with \verb|n| sides and side length \verb|d|.
\end{ex}
\sol{

function drawPolygon(n,d) \{

    for (let i = 0; i < n; i++) \{

        moveForward(d);

        turn(360/n);

    \}
\}
}

\begin{ex}
\begin{multicols}{2}
Write functions \verb|drawSquare(d)| and \verb|drawTriangle(d)| to make the shown image.
\centfig{0.5}{houses}
\end{multicols}
\end{ex}

%\begin{ex}
%Use an if statement to 
%\end{ex}


\newpage
\subsection*{Solutions}
\printcursols
\newpage

\section{More Games}

\begin{ex}
\begin{enumerate}
\item Look at the code in the file \verb|donuts.pdf| on OneNote (or printout) and try to understand what it does.
\item In your 'games' folder create the files \verb|donuts.html|, \verb|donuts.js| and \verb|donuts.css|.
\item Copy the code from \verb|donuts.pdf| into the appropriate files. Make sure to use  the proper styling for your code (best practices). 
\item When you are finished, play the game (open \verb|donuts.html| with LiveServer).
\item Show a different message at the end of the game.
\item Change the speed of the falling donuts and how fast the dog moves around.
\item Experiment with different key combinations to find a layout you like best. You can find the keybindings (number codes for the keys) in the file \verb|keycodes.png|.
\item Can you make the game work with 5 donuts? (You will have to add some more values to the arrays in lines 13, 14 and 15, and change a number on line 23.)
\item See if you can find some other sound effects and use them instead.
\item Create a function that is activated, when a key is released (use \verb|document.onkeyup|).  The function should stop the dog, when you release the arrow key (that means the speed of the dog is set to zero).
\end{enumerate}
\end{ex}


\newpage

\begin{ex}
\begin{enumerate}
\item Fork the repository \url{https://github.com/olidec/donuts}.
\item Clone the files to your personal workspace.
\item Read through the code and try to understand what is going on. Play a game or two.
\item Create a new branch called 'index'
\item Create a file called \verb|index.html| with a short description of the game and a link to the actual game page (\verb|donuts.html|).
\item Merge the new branch back in to your master branch.
\item Create a pull request to my original repository.
\item Create a new branch and give it a sensible name.
\item Change the speed of the falling donuts and how fast the dog moves around.
\item (optional) Create a function that is activated, when a key is released (use \verb|document.onkeyup|).  The function should stop the dog, when you release the arrow key (that means the speed of the dog is set to zero).
\item (optional) Change the ending conditions for the game. How could the game end? Win after score reaches 20? Lose after you miss 3? Be creative.
\item Merge back into the master branch.
\item Create a new branch called 'more-donuts'.
\item Assign a hotkey to add a new donut to the screen, when it is pushed. Use the \verb|.push| method to add to the x, y and speed arrays. (Hint: you will also have to use the length of an array instead of the number $3$ in your code.)
\item Merge back into the master branch.
\item (optional challenge) Rewrite the code so the donuts are objects and x, y and speed are properties of that object.
\item Create a new html file called \verb|arrays.html| and write the answers to the following questions in it as an ordered list. For all of these questions use \verb|a = [2,4,6,8,11]|.
\begin{enumerate}
\item What is the result of \verb|a.length|?
\item What are the entries of \verb|a| after the command \verb|a.push('hello')|?
\item What are the entries of \verb|a| after the command \verb|a.pop()|?
\item What is the output of \verb|a.join(' ** ')| ?
\item What is the output of \verb|a.join()|?
\item What is the output of \verb|a.slice(1,3)|?
\item Use the \verb|forEach()| method to write all the of the array entries into an unordered list.
\end{enumerate}
\item Create a new file called \verb|dom.html| and add a \verb|div| with \verb|id| 'mydiv'.
\begin{enumerate}
\item Write code that will add some text to that div (be creative).
\item Write code that changes the style of the div (again - be creative).
\end{enumerate}
\item Create a pull request to my original repository.
\end{enumerate}
\end{ex}

\newpage


\section{More JavaScript}
In this section we will look at some more Javascript tools and techniques.

\subsection{Accessing the DOM}

The DOM or "Document Object Model" is how Javascript communicates with a webpage.  You can access all elements of a webpage and edit their properties and values. This allows us to make truly interactive websites.

\begin{ex}
Read the documentation on the DOM (w3schools) and/or work through the corresponding section on khanacademy (Links are on OneNote).
\end{ex}

\begin{ex}
Create a website with the examples shown during the lesson: Clock, list of favorite things.
\end{ex}

\begin{ex}
Add a button to your website that changes a CSS property of your choosing.
\end{ex}

\begin{ex}
Add a button and an input field to your website that lets you select the bckground color.
\end{ex}


\newpage

\subsection{Arrays}

An array is a special variable in Javascript that can contain more than one entry. I t can be used as a collection or a list of objects or values:

{\bf Example.}
\begin{verbatim}
let beatles = ['John', 'Paul', 'George', 'Ringo']
\end{verbatim}

We can access individual elements of an array using square brackets and the position number (starting at $0$): 

{\bf Example.}\verb|beatles[2]| results in the output \verb|'George'|.

If you add values that lie outside of the original array,  Javascript will add empty entries to fill up:

{\bf Example.} \verb|beatles[10]='hello'| results in 
\verb|John,Paul,George,Ringo,,,,,,,Hello|

\subsection{Array Methods and Properties}

Functions that modify objects are called \emph{methods}.  \emph{Properties} are Information about the array (mainly length). These are added to the variable itself with a point:

{\bf Example.} \verb|beatles.length| returns the value $4$.

The complete rundown of array methods can be found here:
\url{https://www.w3schools.com/js/js_array_methods.asp}.

The most important ones may be \verb|sort()|, \verb|concat()| and \verb|slice()|. 

\begin{itemize}
\item \verb|sort()| does pretty much what it says on the box. It sorts an array in ascending order. There are of course some technical issues, if the array does not contain numbers, but mostly it works the way, you would expect. (If you want to sort in descendeing order you can use \verb|reverse()| after sorting.
\item \verb|concat()| joins two (or more) arrays together. This does not change the original array -- it creates a new one.
\item \verb|slice()| takes a certain part of the array and places it into a new array.
\end{itemize}



\subsection{Exercises}

\begin{ex}
Change the array to only include the members of the Beatles who are still alive.
\end{ex}

\begin{ex}
Change the array to add your own name somewhere.
\end{ex}
\sol{
beatles[3] = 'Harry Potter'\\
document.getElementById('array-ex2').innerHTML = beatles}

\begin{ex}
Add a value at the fourth position by writing 'beatles[4] = ...' . What happens and why?
\end{ex}
\sol{
beatles[4] = 'Yoko' \\
document.getElementById('array-ex3').innerHTML = beatles}

\begin{ex}
Do the same thing as above, but add the new value at the 10th position.
\end{ex}
\sol{
beatles[10] = 'The new Beatle' \\
document.getElementById('array-ex4').innerHTML = beatles
}

\begin{ex}
Use a string literal to write the following output below: ``The array has length $\_\_\_$'', where Javascript automatically fills in the length.
\end{ex}
\sol{
document.getElementById('array-ex5').innerHTML = `The array has length \$\{ beatles.length \}.`
}

\begin{ex}
Sort the array and show your output. Use this to find the first and last entries alphabetically.
\end{ex}
\sol{
beatles.sort() \\
document.getElementById('array-ex6').innerHTML = beatles \\
console.log(beatles[0]) \\
console.log(beatles[beatles.length-1])
}


\subsection{Solutions}

\printcursols


\newpage

\subsection{More Array Exercises}
\begin{ex}
Regard the following code chunk:
\begin{verbatim}
let myArray = ['Apples', 'Bananas', 'Coconuts', 'Dates']

myArray[2] = 'Cheese'
myArray.push('Crackers')
\end{verbatim}
What are the entries of \verb|myArray| after this code has run?
\end{ex}
\sol{[Apples, Bananas, Cheese, Dates, Crackers]}

\begin{ex}
We now run the code \verb|myArray.sort()|. What are the values of \verb|myArray| now?
\end{ex}
\sol{[Apples, Bananas, Cheese, Crackers, Dates]}

\begin{ex}
What does the following code do (assume that myArray is the same as above)?
\begin{verbatim}
let text = `<ul>`
myArray.forEach(mysteryFunction) 
text += `</ul>`
document.getElementById('solutions').innerHTML = text

function mysteryFunction(string) {
    text += `<li> ${string} </li> `
}
\end{verbatim}
\end{ex}
\sol{
creates a html list with the array entries.
}

\begin{ex}
Given the two arrays
\begin{verbatim}
myFirstArray = [1,2,3]
mySecondArray = ['hat','cat','bat']
\end{verbatim}
What is the difference between \verb|myFirstArray.concat(mySecondArray)| and \verb|mySecondArray.concat(myFirstArray)|?
\end{ex}
\sol{
1,2,3,hat,cat,bat
vs.
hat,cat,bat,1,2,3
}

\begin{ex}
Given the array 

\verb|arrayToJoin = ['Hello','my','name','is','Inigo','Montoya']|

Use the \verb|join| method to get the following output

\verb|Hello ! my ! name ! is ! Inigo ! Montoya|\\
and \\
\verb|Hello /*/ my /*/ name /*/ is /*/ Inigo /*/ Montoya|
\end{ex}
\sol{
arrayToJoin.join(' ! ') \\
arrayToJoin.join(' /*/ ')
}

\begin{ex}
\begin{enumerate}
\item Create a new folder in your workspace.
\item Add files \verb|index.html|, \verb|style.css| and \verb|script.js|
\item Use the basic html template from Visual Studio Code to start your project and save.
\item From your \emph{source control} panel in VSCode, publish your folder to github in a public repository.
\item In your html create a div and give it an id.
\item In your script create an array of your four favorite books. Use the code above to write the titles in a list.
\item Add HTML code with additional information about your 'library'.
\item Style the page using CSS to your liking.
\end{enumerate}
\end{ex}

\begin{ex}
\begin{enumerate}
\item Fork the repository \verb|https://github.com/olidec/array-animation| to your own account.
\item Create a local copy via \verb|Clone|
\item Look at the code and determine what it does.
\item Change the number of images in the animation. Add your own images.
\end{enumerate}
\end{ex}



\newpage


\subsection{Solutions}
\printcursols

\newpage



\newpage

\section{Git and Github}

Git is a version control system.  A version control system allows you to track the changes of your project over time and makes it easier to collaborate with other people on a project.

We have already installed git and created a github account. Github is not directly linked to git: Git is a software that you install on your computer. Github is a website that allows you to store your coding files in so-called online repositories.

Our main interaction with git will be via Visual Studio Code. Once we have linked our copy of VSCode with github, we then have the functionalities of git.

\subsection{The Git Worlflow}
In order to use git, we need to let our software know which folder it should look at. There are two main options:
\begin{itemize}
\item Clone a folder we own from github.
\centfig{0.5}{clone}
\item Publish a folder we created locally to github.
\centfig{0.6}{publish}
\end{itemize}

\newpage

After that our workflow in VSCode uses four main steps: {\bf Edit}, {\bf stage}, {\bf commit} and {\bf publish}.
\begin{itemize}
\item When we edit a document that is linked to github. VSCode will show us in the source control panel.
\centfig{0.4}{source}
\item We then need to stage this edit by clicking the little plus next to it.
\centfig{0.4}{stage}
\item Add a message for your changes.
\centfig{0.4}{message}
\newpage
\item Commit your changes.
\centfig{0.4}{commit}
\item Push your current version to github.
\centfig{0.4}{push}
\end{itemize}

\subsection{Branching}

If we want to work on part of a project without breaking any working parts, we use what are called \emph{branches}. A branch splits off from the main branch and can then be merged back in. Any work on the branch can be saved separately.

\newpage

The easiest way to work with branches is by using the \emph{Git Graph} extension:
\begin{enumerate}
\item Click on the Git Graph button on the bottom left.
\centfig{0.4}{gitgraph}
\item Make sure the master branch is checked out.
\item Right click on the small circle next to the main branch label. and create a new branch. 
\centfig{0.4}{master}
Make sure to select the check out option.
\centfig{0.4}{checkout}
\item Edit your file.
\item Stage, commit and push your edits.
\item Change back to your main branch by double clicking the label of the main branch.
\centfig{0.4}{main}
\item Right click on the label of your edited branch and select the merge option.
\centfig{0.4}{merge}
\end{enumerate}


\newpage


\section{Objects and JSON}

\begin{verbatim}
"In JavaScript, objects are king. If you understand objects, 
you understand JavaScript."
\end{verbatim}

A standard variable can have one value: \verb|x=42|.

Objects are variables that can have many values. We use {\bf name: value} pairs to represent the individual values of an object.  The pairs are then collected in curly brackets and separated by commas. It is common practice to declare objects with the \verb|const| keyword:

\verb|const person = {firstName: 'Arthur', lastName: 'Dent', age: 42}|

You can also define objects over multiple lines. This is sometimes useful to keep a better overview:
\begin{verbatim}
const person = {
	firstName: 'Arthur', 
	lastName: 'Dent', 
	age: 42
	}
\end{verbatim}

\subsection{Object Properties}

All these named values are called \emph{properties}. We can access the individual values with the so-called {\bf dot-notation}: 
\begin{itemize}
\item \verb|person.firstName| returns \emph{Arthur}
\item \verb|person.lasttName| returns \emph{Dent}
\item \verb|person.age| returns \emph{42}
\end{itemize}

\subsection{Object Methods}

\emph{Methods} are actions that can be performed on an object. Methods are properties that contain functions. We can add a \verb|fullName| method to our object:
\begin{verbatim}
const person = {
    firstName: 'Arthur', 
    lastName: 'Dent', 
    age: 42,
    fullName: function() {
        return `${ this.firstName } ${ this.lastName }`
    }
}
\end{verbatim}

This allows us to get the full name of the person saved in the object. Note that if we want to refer to the object itself (or properties of it), we use the keyword \verb|this|. 

When we call a method on an object we always have to call it as a function -- so with brackets, even if it has no arguments: \verb|person.fullName()| will return \emph{Arthur Dent}.

You can even chain methods. So \verb|person.fullName().toUpperCase()| will evaluate to \emph{ARTHUR DENT} (Note: \verb|toUpperCase()| is a predefined method in JavaScript and can be applied to any \emph{string}).

\subsection{JSON} 
JSON stands for {\bf J}ava{\bf S}cript {\bf O}bject {\bf N}otation

JSON is a text format for storing and transporting data

The JSON format is syntactically similar to the code for creating JavaScript objects. Because of this, a JavaScript program can easily convert JSON data into JavaScript objects.

Since the format is text only, JSON data can easily be sent between computers, and used by any programming language.

JavaScript has a built in function for converting JSON strings into JavaScript objects:

\verb|JSON.parse()|

JavaScript also has a built in function for converting an object into a JSON string:

\verb|JSON.stringify()|

So if we want to output our entire object to the DOM, we can use: \verb|JSON.stringify(person)| and we will get the output \emph{\{"firstName":"Arthur","lastName":"Dent","age":42\}}.

\subsection{Important Objects in JavaScript}

\begin{itemize}
\item When an HTML document is loaded into a frame of a browser window, a {\bf Document object} instance — named document — is created for that frame. This document object, like most objects, has a collection of properties and methods. We have often used \verb|document.getElementById(...)|
\item {\bf Strings} are also objects we have some built in methods at our disposal. e.g. \verb|toUpperCase()| etc.
\item The {\bf Math object} provides a number of useful methods and properties for mathematical processing. For example, Math provides the following property: \verb|Math.PI|
\end{itemize}

\subsection{Exercises}
\begin{ex}
Solve the three exercises from w3schools.org

\url{https://www.w3schools.com/js/exercise_js.asp?filename=exercise_js_objects1}


\end{ex}

\begin{ex}
Create an object to hold information on your favorite recipe. It should have properties for title (a string), servings (a number), and ingredients (an array of strings).
On separate lines (use \verb|divs| and/or html lists), log the recipe information so it looks like:
\begin{itemize}
\item Mole
\item Serves: 2
\item Ingredients:
\item cinnamon
\item cumin
\item cocoa
\end{itemize}

\end{ex}

\begin{ex}
Create an array of objects, where each object describes a book and has properties for the title (a string), author (a string), and alreadyRead (a boolean indicating if you read it yet).

Iterate through the array of books. For each book, write the book title and book author to an output \verb|div| like so: "The Hobbit by J.R.R. Tolkien".

Now use an if/else statement to change the output depending on whether you read it yet or not. If you read it, log a string like 'You already read "The Hobbit" by J.R.R. Tolkien', and if not, log a string like 'You still need to read "The Lord of the Rings" by J.R.R. Tolkien.'
\end{ex}












\end{document}
