\documentclass[11pt,a4paper]{report}

\usepackage{xcolor}
\def\farbe{cyan}

\usepackage{dclecture}

\usepackage{listings}
\lstset{language=Python}


\definecolor{codegreen}{rgb}{0,0.6,0}
\definecolor{codegray}{rgb}{0.9,0.9,0.9}
\definecolor{codepurple}{rgb}{0.58,0,0.82}
\definecolor{backcolour}{rgb}{0.95,0.95,0.92}

\lstdefinestyle{mystyle}{
%	morekeywords={forward,turn},
    backgroundcolor=\color{codegray},
    commentstyle=\color{codegreen},
%    keywordstyle=\color{codegreen},
    numberstyle=\tiny\color{gray},
    stringstyle=\color{codepurple},
    basicstyle=\footnotesize,
    identifierstyle=\color{blue},
    stringstyle=\color{orange},
    breakatwhitespace=false,
    breaklines=true,
    captionpos=b,
    keepspaces=true,
    numbers=left,
    numbersep=5pt,
    showspaces=false,
    showstringspaces=false,
    showtabs=false,
    tabsize=2
}

\lstset{style=mystyle}




%%% Fancy Header and Footer
\renewcommand{\headrule}{\vbox to 0pt{\hbox to\headwidth{\color{\farbe}\rule{\headwidth}{1pt}}\vss}}
\pagestyle{fancy} %eigener Seitenstil
\fancyhf{} %alle Kopf- und Fusszeilenfelder bereinigen
\fancyhead[C]{Computer Science} %Kopfzeile mitte
%\fancyhead[R]{\includegraphics[width=0.2cm]{x.png}}
\fancyfoot[C]{\thepage}







\begin{document}
\section{Cryptography}

\subsection{Substitution Ciphers}

\begin{ex}
Encrypt \verb|THE QUICK BROWN FOX| using a Caesar shift of 17 by hand.
\end{ex}


\begin{ex}
Eve intercepted this message: \verb|N QNPJ HNUMJWX.| Figure out how to break it to get Alice’s message.
\end{ex}


\begin{ex}
Use the Caesar Decryption tool on the Jupyter notebook to decode the following message:
\begin{verbatim}
Pyebc mybok xncof oxiok bckqy yebpk drobc lbyeq rdpyb drezy xdrsc
myxds xoxd, kxogx kdsyx ,myxm osfon sxVsl obdi, kxnno nsmkd ondyd
rozby zycsd syxdr kdkvv woxkb ombok donoa ekv.
\end{verbatim}
\end{ex}

\begin{ex}
Alice tries to make the Caesar code better by using multiple Caesar shifts. Her argument is: "If I use 50 Caesar shifts,  then the keyspace will be $26^{50}$ -- which is much larger than the standard keyspace of 26 for the Caesar code.  This means the code will be harder to crack." 

Is her argument sound?
\end{ex}

\begin{ex}
Given the following One Time Pad: 
\begin{verbatim}
74288, 12127, 11380, 68612, 71750, 80865, 45593, 84621, 37271,
22781, 46647, 18730, 62697, 23437, 36413, 51216, 65713, 74749,
44834, 89655
\end{verbatim}
encrypt the message \verb|TOP  SECRET| using a standard alphabet replacement (a = 1, b = 2, etc.) by hand.
\end{ex}

\begin{ex}
Given the following One Time Pad: 
\begin{verbatim}
74288, 12127, 11380, 68612, 71750, 80865, 45593, 84621, 37271,
22781, 46647, 18730, 62697, 23437, 36413, 51216, 65713, 74749,
44834, 89655
\end{verbatim}
encrypt the message \verb|TOP  SECRET| using the ASCII table by hand.
\end{ex}

\begin{ex}
Using the code from the Caesar Code as a guide, try to write code that wiill encrypt text using a given  One Time Pad.
\end{ex}






\begin{ex}
Encrypt the following text using a Vigenère Cipher with the keyword \verb|SUPERMAN| by hand:
\begin{verbatim}
I LOVE MATH
\end{verbatim}

\end{ex}


\begin{ex}
Encrypt the following text using a Vigenère Cipher with the keyword \verb|SUPERMAN| using the tool on the Jupyter notebook:
\begin{verbatim}
To be, or not to be, that is the question:
Whether 'tis nobler in the mind to suffer
The slings and arrows of outrageous fortune,
Or to take arms against a sea of troubles
And by opposing end them. To die - to sleep,
No more; and by a sleep to say we end
The heart-ache and the thousand natural shocks
That flesh is heir to: 'tis a consummation
Devoutly to be wish'd. To die, to sleep;
\end{verbatim}
\end{ex}


\begin{ex}
Using only the code from the Caesar Code, try to write code that wiill encrypt text using the Vigenère method.
\end{ex}


\subsection{Transposition Ciphers}


\begin{ex}
Write out the message
\verb|BEWARE THE IDES OF MARCH.|
in rows of length 5 and read out the columns. Give the ciphertext.
\end{ex}

\begin{ex}
Decrypt the following message using length 4: \verb|INLRREINVCFIIAOA|.
\end{ex}

\begin{ex}
Eve intercepted the following message. Help her break it: \verb|HYDAMAPOYPNZ|.
\end{ex}





\subsection{Modular Arithmetic}

\begin{ex}
What is $51\mod 4$ ?
\end{ex}

\begin{ex}
What is $(500 + 160) \mod 2$ ?
\end{ex}


\begin{ex}
What are the only possible solutions for $x\mod 4$ ?
\end{ex}


\begin{ex}
In general, what are the possible solutions for $x\mod n$ ?
\end{ex}


\subsubsection{Multiplying with modular arithmetic}

Multiplying also works in the natural way. We multiply the solutions to $x = a\mod b$ and $y = c \mod b$ and then we reduce $(x\cdot y) \mod b$.
Dividing is harder. Using when dividing by $n$, we multiply by the reciprocal $\frac{1}{n}=n^{-1}$. But since we are only working with natural numbers, the fraction $\frac{1}{n}$ doesn’t exist. So we need a new way to
define a multiplicative inverse:
$a^{-1}$ is the number less than $b$ that satisfies $(a^{-1}\cdot a) \mod b = 1$


\begin{ex}
What is the multiplicative inverse of $5\mod 7$ ?
\end{ex}

\begin{ex}
What is the multiplicative inverse of $4\mod 9$ ?
\end{ex}

\begin{ex}
Does $2$ have a multiplicative inverse $\mod 4$ ?
\end{ex}

\end{document}
