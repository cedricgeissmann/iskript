\documentclass[11pt,a4paper]{report}

\usepackage{xcolor}
\def\farbe{cyan}

\usepackage{dclecture}

\usepackage{listings}
\lstset{language=Python}  


\definecolor{codegreen}{rgb}{0,100,0}
\definecolor{codegray}{rgb}{0.5,0.5,0.5}
\definecolor{codepurple}{rgb}{0.58,0,0.82}
\definecolor{backcolour}{rgb}{0.95,0.95,0.92}

\lstdefinestyle{mystyle}{
%    backgroundcolor=\color{backcolour},   
%    commentstyle=\color{codegreen},
    keywordstyle=\color{codegreen},
    numberstyle=\tiny\color{codegray},
%    stringstyle=\color{codepurple},
    basicstyle=\ttfamily\footnotesize,
    breakatwhitespace=false,         
    breaklines=true,                 
    captionpos=b,                    
    keepspaces=true,                 
    numbers=left,                    
    numbersep=5pt,                  
    showspaces=false,                
    showstringspaces=false,
    showtabs=false,                  
    tabsize=2
}

\lstset{style=mystyle}




%%% Fancy Header and Footer
\renewcommand{\headrule}{\vbox to 0pt{\hbox to\headwidth{\color{\farbe}\rule{\headwidth}{1pt}}\vss}}
\pagestyle{fancy} %eigener Seitenstil
\fancyhf{} %alle Kopf- und Fusszeilenfelder bereinigen
\fancyhead[C]{Computer Science} %Kopfzeile mitte
%\fancyhead[R]{\includegraphics[width=0.2cm]{x.png}}
\fancyfoot[C]{\thepage}







\begin{document}
\section{Turtle Graphics}




\subsection{Basics}

\begin{ex}
Write a program that draws a square with sides 150.
\end{ex}

\begin{ex}
Write a program that draws the following parallelogram.
\centfig{0.5}{parallelogram}
\end{ex}

\begin{ex}
Write a program that draws a lightning bolt. The exact  dimensions can be chosen freely.
\centfig{0.1}{arrow}
\end{ex}


\begin{ex}
Draw the following house without going backwards or moving along the same line twice. Hint: Use lengths 185 for the base and 262 for the long diagonals.
\centfig{0.4}{nikolaus}
\end{ex}




\newpage
\subsection{Functions}
\begin{ex}
Write a function that draws a regular polygon where the number of sides and the length of the sides are parameters.
\end{ex}


\begin{ex}
What does the following function do?
\begin{lstlisting}
def jump(n):
    pen_up()
    forward(n)
    pen_down() \end{lstlisting}
\end{ex}


\begin{ex}
Write a \verb|jump_to| function that places the turtle at any starting position in the graphics frame.
\end{ex}




\newpage

\subsection{Loops}

\begin{ex}
The following code draws a circle.
\begin{lstlisting}
from gymmu.turtle import *

make_turtle()

def draw_circle():
    for i in range(360):
        forward(1)
        turn(1)

draw_circle()
    
show()
\end{lstlisting}

\begin{enumerate}
\item Think about why this draws a circle.
\item can you add a square around this circle?
\end{enumerate}
\end{ex}

\subsection{Expand your Horizon}


\end{document}