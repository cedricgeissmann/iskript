\documentclass[11pt,a4paper]{report}

\usepackage{xcolor}
\def\farbe{cyan}

\usepackage{dclecture}

\usepackage{listings}
\lstset{language=Python}


\definecolor{codegreen}{rgb}{0,0.6,0}
\definecolor{codegray}{rgb}{0.9,0.9,0.9}
\definecolor{codepurple}{rgb}{0.58,0,0.82}
\definecolor{backcolour}{rgb}{0.95,0.95,0.92}

\lstdefinestyle{mystyle}{
%	morekeywords={forward,turn},
    backgroundcolor=\color{codegray},
    commentstyle=\color{codegreen},
%    keywordstyle=\color{codegreen},
    numberstyle=\tiny\color{gray},
    stringstyle=\color{codepurple},
    basicstyle=\footnotesize,
    identifierstyle=\color{blue},
    stringstyle=\color{orange},
    breakatwhitespace=false,
    breaklines=true,
    captionpos=b,
    keepspaces=true,
    numbers=left,
    numbersep=5pt,
    showspaces=false,
    showstringspaces=false,
    showtabs=false,
    tabsize=2
}

\lstset{style=mystyle}




%%% Fancy Header and Footer
\renewcommand{\headrule}{\vbox to 0pt{\hbox to\headwidth{\color{\farbe}\rule{\headwidth}{1pt}}\vss}}
\pagestyle{fancy} %eigener Seitenstil
\fancyhf{} %alle Kopf- und Fusszeilenfelder bereinigen
\fancyhead[C]{Computer Science} %Kopfzeile mitte
%\fancyhead[R]{\includegraphics[width=0.2cm]{x.png}}
\fancyfoot[C]{\thepage}







\begin{document}
\section{Turtle Graphics}




\subsection{Basics}

\begin{ex}
Write a program that draws a square with sides 150.
\end{ex}

\begin{ex}
Write a program that draws the following parallelogram.
\centfig{0.5}{parallelogram}
\end{ex}

\begin{ex}
What does the following code draw? Try to solve it in your head first before you copy and paste
\begin{lstlisting}
from gymmu.turtle import *

make_turtle()

forward(80)
turn(90)
forward(80)
turn(-90)
forward(80)
turn(90)
forward(80)
turn(-90)
forward(80)

show()
\end{lstlisting}
\end{ex}

\newpage

\begin{ex}
What does the following code draw? Try to solve it in your head first before you copy and paste
\begin{lstlisting}
from gymmu.turtle import *

make_turtle()

forward(80)
turn(90)
forward(80)
turn(-90)
forward(-80)
turn(-90)
forward(-80)
turn(90)
forward(80)

show()
\end{lstlisting}
\end{ex}

\begin{ex}
Write a program that draws a lightning bolt. The exact  dimensions can be chosen freely.
\centfig{0.1}{arrow}
\end{ex}


\newpage

\begin{ex}
Draw the following house without going backwards or moving along the same line twice. Hint: Use lengths 185 for the base and 262 for the long diagonals.
\centfig{0.4}{nikolaus}
\end{ex}




\newpage
\subsection{Functions}
\begin{ex}
Write a function that draws a regular polygon where the number of sides and the length of the sides are parameters.
\end{ex}


\begin{ex}
What does the following function do?
\begin{lstlisting}
from gymmu.turtle import *

make_turtle()

def jump(n):
    pen_up()
    forward(n)
    pen_down()

	show()
\end{lstlisting}
\end{ex}


\begin{ex}
Write a \verb|jump_to| function that places the turtle at any starting position in the graphics frame.
\end{ex}




\newpage

\subsection{Loops}


\begin{ex}
Write a function that draws ten small squares in a row.
\end{ex}

\begin{ex}
Draw the following spiral:
\centfig{0.5}{square_spiral}
\end{ex}

\begin{ex}
Starting with a function that draws a square, can you write a new function that draws a filled square?
\begin{lstlisting}
from gymmu.turtle import *

make_turtle()

def draw_square(a):
    for i in range(4):
        forward(a)
        turn(90)

show()
\end{lstlisting}
\end{ex}

\newpage


\begin{ex}
The following code draws a circle.
\begin{lstlisting}
from gymmu.turtle import *

make_turtle()

def draw_circle():
    for i in range(360):
        forward(1)
        turn(1)

draw_circle()

show()
\end{lstlisting}

\begin{enumerate}
\item Think about why this draws a circle.
\item Can you add a square around this circle?
\end{enumerate}
\end{ex}



\newpage

\subsection{Expand your Horizon}

\begin{ex}
The following code draws a five-pointed star.
\begin{lstlisting}
from gymmu.turtle import *

make_turtle()

for i in range(5):
    forward(200)
    turn(-144)

show()
\end{lstlisting}
\begin{enumerate}
\item Can you change the code that it draws a nine-pointed star? How about a six-pointed star? Could you make as many points as you want?
\item Can you change the code to create this image:
\centfig{0.5}{star_spiral}
\end{enumerate}
\end{ex}

\newpage

\begin{ex}
Run the followinig code in a code block:
\begin{lstlisting}
from gymmu.turtle import *

colors = ['red', 'purple', 'blue', 'green']

make_turtle()

for x in range(360):
    color(colors[x%4])
    stroke_width(x/200 + 1)
    forward(x)
    turn(89)

show()
\end{lstlisting}
Can you change the code to produce the following image?
\centfig{0.5}{color_spiral}
\end{ex}

\begin{ex}
Given is the following code.
\begin{lstlisting}
from gymmu.turtle import *

make_turtle()

forward(100)
turn(90)
forward(100)
turn(90)
forward(100)
turn(90)

show()
\end{lstlisting}
Does the code produce this image? If not, can you fix it?
\centfig{0.5}{error_1}
\end{ex}

\begin{ex}
Given is the following code.
\begin{lstlisting}
from gymmu.turtle import *

make_turtle()

forward(100)
turn(90)
forward(100)
turn(90)
forward(100)
turn(90)
forward(100)
\end{lstlisting}
Does the code produce this image? If not, can you fix it?
\centfig{0.5}{error_1}
\end{ex}


\begin{ex}
Given is the following code.
\begin{lstlisting}
from gymmu.turtle import *

make_turtle()

forward(100)
turn(90)
forward(100)
turn(90)
forward(100)
turn(90)
forward(100)
turn(90)
forward(100)

show()
\end{lstlisting}
Does the code produce this image? If not, can you fix it?

\textbf{Hint:} Look closely at the position of the turtle.
\centfig{0.5}{error_1}
\end{ex}


\begin{ex}
Given is the following code.
\begin{lstlisting}
from gymmu.turtle import *

make_turtle()

for i in range(2):
    turn(90)
    forward(100)

show()
\end{lstlisting}
Does the code produce this image? If not, can you fix it?
\centfig{0.5}{error_2}
\end{ex}

\begin{ex}
Given is the following code.
\begin{lstlisting}
from gymmu.turtle import *

make_turtle()

for i in range(3):
    forward(100)
    turn(90)

show()
\end{lstlisting}
Does the code produce this image? If not, can you fix it?

\textbf{Hint:} Look closely at the position of the turtle.
\centfig{0.5}{error_2}
\end{ex}

\begin{ex}
Given is the following code.
\begin{lstlisting}
from gymmu.turtle import *

make_turtle()

for i in range(10):
    turn(90)
    forward(10 + i)

show()
\end{lstlisting}
Does the code produce this image? If not, can you fix it?
\centfig{0.5}{error_3}
\end{ex}


\begin{ex}
Given is the following code.
\begin{lstlisting}
from gymmu.turtle import *

make_turtle()

for i in range(10):
    turn(90)
    forward(10 * (i + 1))

show()
\end{lstlisting}
Does the code produce this image? If not, can you fix it?
\centfig{0.5}{error_3}
\end{ex}

\end{document}
