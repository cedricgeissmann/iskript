\documentclass[11pt,a4paper]{report}

\usepackage{xcolor}
\def\farbe{cyan}

\usepackage{dclecture}

\usepackage{listings}
\lstset{language=Python}


\definecolor{codegreen}{rgb}{0,0.6,0}
\definecolor{codegray}{rgb}{0.9,0.9,0.9}
\definecolor{codepurple}{rgb}{0.58,0,0.82}
\definecolor{backcolour}{rgb}{0.95,0.95,0.92}

\lstdefinestyle{mystyle}{
%	morekeywords={forward,turn},
    backgroundcolor=\color{codegray},
    commentstyle=\color{codegreen},
%    keywordstyle=\color{codegreen},
    numberstyle=\tiny\color{white},
    stringstyle=\color{codepurple},
    basicstyle=\footnotesize,
    identifierstyle=\color{blue},
    stringstyle=\color{orange},
    breakatwhitespace=false,
    breaklines=true,
    captionpos=b,
    keepspaces=true,
    numbers=left,
    numbersep=5pt,
    showspaces=false,
    showstringspaces=false,
    showtabs=false,
    tabsize=2
}

\lstset{style=mystyle}




%%% Fancy Header and Footer
\renewcommand{\headrule}{\vbox to 0pt{\hbox to\headwidth{\color{\farbe}\rule{\headwidth}{1pt}}\vss}}
\pagestyle{fancy} %eigener Seitenstil
\fancyhf{} %alle Kopf- und Fusszeilenfelder bereinigen
\fancyhead[C]{Computer Science} %Kopfzeile mitte
%\fancyhead[R]{\includegraphics[width=0.2cm]{x.png}}
\fancyfoot[C]{\thepage}



\usepackage{glossaries}

\makeglossaries

\newcommand*{\textbfblue}[1]{\textbf{\color{blue}{#1}}}
\renewcommand*{\glstextformat}[1]{\textbfblue{#1}}


\newglossaryentry{module}
{
  name=module,
  description={a text file containing definitions and functions that can be used once imported.}
}


\begin{document}
\section{Summary}


\subsection{The Basics of the Turtle Environment}

\begin{itemize}
\item \begin{lstlisting}
from gymmu.turtle import *
\end{lstlisting}

This imports the turtle \gls{module} containing all the commands we use to manipulate the turtle. The star * means that everything is imported.

\item \begin{lstlisting}
make_turtle()
\end{lstlisting}

Creates a blank working space and resets the turtle to position (200,200) and facing to the right.



\item \begin{lstlisting}
show()
\end{lstlisting}

Draws the resulting image from the preceding commands and shows the position of the turtle.

\end{itemize}


\subsection{Turtle Commands}

\begin{itemize}
\item \begin{lstlisting}
forward(n)
\end{lstlisting}

Moves the turtle \texttt{n} \emph{units} forward.

\item \begin{lstlisting}
turn(a)
\end{lstlisting}

Rotates the turtle \texttt{a} degrees counter clockwise.

\end{itemize}

\subsection{Python Commands}

\item \begin{lstlisting}
for i in range(n):
	the code
\end{lstlisting}

Repeats \texttt{the code} \texttt{n} times. Creates a \gsl{counter} named \texttt{i} that has each value from $0$ to $n-1$ in 



\printglossary

\end{document}
