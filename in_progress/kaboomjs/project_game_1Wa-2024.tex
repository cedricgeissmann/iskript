\documentclass[11pt,a4paper]{report}

\usepackage{xcolor}
\usepackage[ngerman]{babel}
\def\farbe{blue}

\usepackage{dclecture}

\setlength{\headheight}{24pt}

\usepackage{listings}
\lstset{language=Python}


\definecolor{codegreen}{rgb}{0,0.6,0}
\definecolor{codegray}{rgb}{0.9,0.9,0.9}
\definecolor{codepurple}{rgb}{0.58,0,0.82}
\definecolor{backcolour}{rgb}{0.95,0.95,0.92}

\lstdefinestyle{mystyle}{
%	morekeywords={forward,turn},
    backgroundcolor=\color{codegray},
    commentstyle=\color{codegreen},
%    keywordstyle=\color{codegreen},
    numberstyle=\tiny\color{gray},
    stringstyle=\color{codepurple},
    basicstyle=\footnotesize,
    identifierstyle=\color{blue},
    stringstyle=\color{orange},
    breakatwhitespace=false,
    breaklines=true,
    captionpos=b,
    keepspaces=true,
    numbers=left,
    numbersep=5pt,
    showspaces=false,
    showstringspaces=false,
    showtabs=false,
    tabsize=2
}

\lstset{style=mystyle}




%%% Fancy Header and Footer
\renewcommand{\headrule}{\vbox to 0pt{\hbox to\headwidth{\color{\farbe}\rule{\headwidth}{1pt}}\vss}}
\pagestyle{fancy} %eigener Seitenstil
\fancyhf{} %alle Kopf- und Fusszeilenfelder bereinigen
\fancyhead[C]{Informatik - Gruppenarbeit - Spiel Projekt} %Kopfzeile mitte
%\fancyhead[R]{\includegraphics[width=0.2cm]{x.png}}
\fancyfoot[C]{\thepage}


\title{Informatik Gruppenarbeit 1Wa}
\date{Winter/Frühjahr 2024}
% \author{DC}



\begin{document}
\maketitle

\newpage
\section{Erste Schritte}

In den nächsten Wochen werden Sie in Gruppen an einem Programmierprojekt
arbeiten. Sie werden dabei ein interaktives Spiel entwickeln. Dabei werden Sie
das Projekt das wir im Unterricht begonnen haben anpassen und erweitern, um ein
tolles kreatives Spiel zu erstellen.

\subsection{Gruppen}

Im ersten Schritt werden Sie Gruppen bilden. Eine Gruppe besteht aus 3 bis 4
Personen. Die Gruppen müssen nicht nach Praktikum eingeteilt sein, es kann
sogar Sinn ergeben wenn die Gruppe auf verschiedene Praktikumsgruppen verteilt
ist, damit Sie mehr Zeit für Fragen haben.

% \subsection{Installieren der Werkzeuge}

% Um an dem Projekt zu arbeiten müssen Sie einige Werkzeuge bei Ihnen lokal auf
% dem Computer installieren. Einige Werkzeuge haben Sie vermutlich bereits
% installiert. Einige Werkzeuge brauchen Sie nicht unbedingt, können es Ihnen
% aber einfacher machen.

% \begin{itemize}
% \item Damit Sie Versionskontrolle für die Zusammenarbeit nutzen können, und
%     damit überprüft werden kann wer wieviel beigetragen hat, müssen Sie
%     \verb|Git| lokal auf Ihrem Computer installieren (dies haben Sie bereits
%     gemacht).

%     \begin{center}
%         \url{https://git-scm.com/downloads}
%     \end{center}

% \item Als Programmierumgebung verwenden wir \verb|Visual Studio Code|. Das
%     haben Sie bereits installiert.

%         \begin{center}
%             \url{https://code.visualstudio.com/}
%         \end{center}

% \item Damit Sie in \verb|Visual Studio Code| so arbeiten können wie Sie sich
%     das gewohnt sind, müssen Sie noch Extensions installieren:
% \begin{itemize}
% \item Git Graph (das von \emph{mhutchie})
% \item Live Server
% \item Prettier (Automatisches formatieren von Code)

% \end{itemize}

% \end{itemize}

\newpage

\section{Projektbeschreibung}

\subsection{Ziele} 

Ziel des Projektes ist es ein interaktives Spiel mit standard Webtechnologien
(HTML, CSS und Javascript) zu entwickeln. Der Fokus dabei liegt auf der
Entwicklung eines Einzelspielerspiels. Mehrspielerspiele können auch entwickelt
werden, diese sollten aber am gleichen Computer stattfinden.

Einige Beispiele zur Inspiration:
\begin{itemize}
\item Jump \& Run
\item Top-Down RPG
\item Rundenbasiertes Karten oder Brettspiel
\item Capture the Flag
\item Worms
\item Lemmings
\item etc.
\end{itemize}

\subsection{Ablauf}
Das Projekt dauert von der Woche 6 bis zur Woche 15. Die Abschlusspräsentation
mit Demo des Spiels wird am 16.04.2024 stattfinden. Die erste Woche sollte dafür
verwendet werden einen Plan für das Spiel zu erstellen und sich mit
nochmals mit dem Projekt vertraut zu machen. Dabei sollten Sie unbedingt eine
Liste mit Anforderungen machen, was Sie alles an Spiellogik brauchen. Ungefähr
in der Hälfte des Projektes sollten Sie eine minimale Version von Ihrem Spiel
haben, die bereits einige Basisfeatures implementiert. Die finale Version des
Spiel muss spielbar sein.

Für die Dokumentation des Spiels (worum geht es, wie wird es gespielt, wer hat
es entwickelt, wo ist es verfügbar) schreiben Sie eine eigene Webseite (diese
können Sie via Github - Pages veröffentlichen).

\subsubsection{Milensteine}
\begin{description}
    \item[am 27.02.2024] Eine kurze Präsentation (5 Minuten) über die Idee
        Ihres Spiels. Die Klasse wird da die Gelegenheit haben Fragen zu
        stellen und Anregungen zu geben.
\item[um den 12.03.2024] In einem persönlichen Treffen stellen Sie mir Ihr
    Projekt vor, was Sie bis jetzt gemacht haben und woran Sie noch arbeiten
        werden.
    \item[am 15.04.2024] Finale Version (inklusive Webseite) muss via Github
        abgegeben werden. Es findet eine Präsentation mit Demo vor der Klasse
        statt. Auftretende Fragen zum Projekt sollen beantwortet werden.
\end{description}

\subsection{Benotung}

Sie werden eine Note als Gruppe für das Projekt erhalten. Faktoren die bei der
Benotung berücksichtigt werden sind: Haben Sie die Zeit sinnvoll eingeteilt?
Hat jeder in der Gruppe zum Projekt beigetragen? Machen Sie Fortschritt? Wie
wurde das Feedback von Meilenstein 2 umgesetzt? Funktioniert Ihr Spiel? Gibt es
Bugs oder nicht fertige Features? Ist die Dokumentation vollständig und
verständlich? \dots

Die Note ist für alle Gruppenmitglieder die gleiche. Treten jedoch starke
Abweichungen in den Beiträgen auf (git commits) kann die Note für einzelne
Gruppenmitglieder separat gesetzt werden.

\newpage

\subsection{Ressourcen}

Als Hauptressource gilt das Internet und alles was wir bisher im Unterricht
gemacht haben. Wenn Sie Code aus dem Internet verwenden, stellen Sie sicher das
Sie den Code verstehen, und halten Sie fest woher der Code stammt.

Bei Fragen und Problemen können Sie mich natürlich jederzeit kontaktieren.

\end{document}
